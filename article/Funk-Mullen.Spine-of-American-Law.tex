\documentclass[12pt,]{article}
\usepackage[]{mathpazo}
\usepackage{amssymb,amsmath}
\usepackage{ifxetex,ifluatex}
\usepackage{fixltx2e} % provides \textsubscript
\ifnum 0\ifxetex 1\fi\ifluatex 1\fi=0 % if pdftex
  \usepackage[T1]{fontenc}
  \usepackage[utf8]{inputenc}
\else % if luatex or xelatex
  \ifxetex
    \usepackage{mathspec}
  \else
    \usepackage{fontspec}
  \fi
  \defaultfontfeatures{Ligatures=TeX,Scale=MatchLowercase}
\fi
% use upquote if available, for straight quotes in verbatim environments
\IfFileExists{upquote.sty}{\usepackage{upquote}}{}
% use microtype if available
\IfFileExists{microtype.sty}{%
\usepackage{microtype}
\UseMicrotypeSet[protrusion]{basicmath} % disable protrusion for tt fonts
}{}
\usepackage[margin=1in]{geometry}
\usepackage{hyperref}
\hypersetup{unicode=true,
            pdfborder={0 0 0},
            breaklinks=true}
\urlstyle{same}  % don't use monospace font for urls
\usepackage{longtable,booktabs}
\usepackage{graphicx,grffile}
\makeatletter
\def\maxwidth{\ifdim\Gin@nat@width>\linewidth\linewidth\else\Gin@nat@width\fi}
\def\maxheight{\ifdim\Gin@nat@height>\textheight\textheight\else\Gin@nat@height\fi}
\makeatother
% Scale images if necessary, so that they will not overflow the page
% margins by default, and it is still possible to overwrite the defaults
% using explicit options in \includegraphics[width, height, ...]{}
\setkeys{Gin}{width=\maxwidth,height=\maxheight,keepaspectratio}
\IfFileExists{parskip.sty}{%
\usepackage{parskip}
}{% else
\setlength{\parindent}{0pt}
\setlength{\parskip}{6pt plus 2pt minus 1pt}
}
\setlength{\emergencystretch}{3em}  % prevent overfull lines
\providecommand{\tightlist}{%
  \setlength{\itemsep}{0pt}\setlength{\parskip}{0pt}}
\setcounter{secnumdepth}{0}
% Redefines (sub)paragraphs to behave more like sections
\ifx\paragraph\undefined\else
\let\oldparagraph\paragraph
\renewcommand{\paragraph}[1]{\oldparagraph{#1}\mbox{}}
\fi
\ifx\subparagraph\undefined\else
\let\oldsubparagraph\subparagraph
\renewcommand{\subparagraph}[1]{\oldsubparagraph{#1}\mbox{}}
\fi

%%% Use protect on footnotes to avoid problems with footnotes in titles
\let\rmarkdownfootnote\footnote%
\def\footnote{\protect\rmarkdownfootnote}

%%% Change title format to be more compact
\usepackage{titling}

% Create subtitle command for use in maketitle
\newcommand{\subtitle}[1]{
  \posttitle{
    \begin{center}\large#1\end{center}
    }
}

\setlength{\droptitle}{-2em}
  \title{The Spine of American Law:\\
Digital Text Analysis and U.S. Legal Practice}
  \pretitle{\vspace{\droptitle}\Large}
  \posttitle{\par\vspace{0.2in}}
  % \author{true \\ true}
  \author{  \normalsize Kellen Funk \\ \small \emph{Princeton University} \\ \small \href{mailto:kfunk@princeton.edu}{\nolinkurl{kfunk@princeton.edu}}  \\ \\   \normalsize Lincoln Mullen \\ \small \emph{George Mason University} \\ \small \href{mailto:lincoln@lincolnmullen.com}{\nolinkurl{lincoln@lincolnmullen.com}} }
  % \author{\Large Kellen Funk\vspace{0.05in} \normalsize\emph{Princeton University} \footnotesize \url{\href{mailto:kfunk@princeton.edu}{\nolinkurl{kfunk@princeton.edu}}}\vspace*{0.2in}  \and \Large Lincoln Mullen\vspace{0.05in} \normalsize\emph{George Mason University} \footnotesize \url{\href{mailto:lincoln@lincolnmullen.com}{\nolinkurl{lincoln@lincolnmullen.com}}}\vspace*{0.2in} }
  \preauthor{}
  \postauthor{\par}
  \date{}
  \predate{}\postdate{}

\begin{document}
\maketitle

~

At the opening of the first Nevada legislature in 1861, Territorial
Governor James W. Nye, a former New York lawyer, instructed the assembly
that they would have to forsake their inherited Mormon statutes that
were ill adapted to ``the mining interests'' of the new territory.
``Happily for us, a neighboring State whose interests are similar to
ours, has established a code of laws'' sufficiently attractive to
``capital from abroad.'' That neighbor was California, and Nye urged
that California's ``Practice Code'' be enacted in Nevada, as far as it
could ``be made applicable.''\footnote{Message of the Governor, in
  Journal of the Council for the Territory of Nevada (1862), 21.}
Territorial Senator William Morris Stewart, a famed mining lawyer who
would lead the U.S. Senate during Reconstruction, followed the
instructions perhaps too well. Stewart literally cut and pasted the
latest \emph{Wood's Digest} of the California Practice Act into the
session bill, crossing out \emph{state} and \emph{California} and
substituting \emph{territory} and \emph{Nevada} where necessary. Stewart
copied not just California's procedure code but also its method of
codification, for California had in turn borrowed its code by modifying
New York's.

\begin{figure}
\centering
\includegraphics{CB021-1861.jpg}
\caption{Detail from Council Bill 21, First Territorial Legislative
Session (1861), Nevada State Library, Archives and Public Records.}
\end{figure}

Nye wrote back to the assembly in disgust. The bill---of 715
sections---had reached him late the night before the legislative session
was to close. Even in the few hours he had to read it, Nye counted
``many errors in the enrolling of it, numbering probably more than three
hundred.'' Some errors were severe. The code overwrote the specific
jurisdictional boundaries of
Nevada\textless80\textgreater\textless99\textgreater s Organic Act by
copying California\textless80\textgreater\textless99\textgreater s
arrangements. Error-riddled and unconstitutional as the bill was, Nye
believed a civil practice code was a ``universal necessity and public
need,'' doubting ``whether your courts would be able to fulfill the
purpose of their creation'' without one.\footnote{Nevada Council Journal
  (1862), 261--62. Act of Congress (1861) Organizing the Territory of
  Nevada, 12 U.S. Statutes at Large 209--14 (1863). 1861 Nevada Laws
  314.} Nye signed the code into law.

Nothing like this ``universal necessity'' existed when Nye began his
legal career in New York in the 1840s; rather, it was one of the central
developments of America law after the mid-nineteenth century. By 1900,
thirty-one American states and territories had adapted the text of a New
York code of civil procedure first promulgated in 1848. The code became
known as the Field Code, after its chief draftsman David Dudley Field, a
Manhattan trial lawyer.\footnote{Besides his work on codification, which
  extended to civil, penal, and even international codes of law, Field
  became renown for his trial advocacy. Field argued the winning side in
  major Reconstruction cases such as \emph{Ex Parte Milligan}, 71 U.S.
  (4 Wall.) 2 (1866) (holding the trial of civilians by military
  commission unconstitutional), \emph{Cummings v. Missouri}, 71 U.S. (4
  Wall.) 277 (1867) (striking a loyalty oath as unconstitutional), and
  \emph{United States v. Cruikshank}, 92 U.S. 542 (1875) (enforcing the
  Fourteenth Amendment only against ``state action''). Field came under
  heavy censure for his representation of Gilded Age robber barons like
  Jay Gould, Jim Fisk, and William ``Boss'' Tweed, but even Tweed's
  chief adversary Samuel Tilden retained Field's services for the
  disputed election of 1876. See Henry Martyn Field, \emph{The Life of
  David Dudley Field} (1898); Philip J. Bergan, ``David Dudley Field: A
  Lawyer's Life,'' in \emph{The Fields and the Law} (Federal Bar
  Council, 1986).} When Field's code appeared in the Colorado assembly
in 1876, a Denver newspaper wryly commented, ``The scissors and
paste-pot we had heretofore confidently believed were implements
peculiar to the newspaper sanctum.'' To the editor, the cut-and-paste
code was not just a curiosity of legislative history. Rather, the
extensive copying of Field's Code threatened the foundations of American
popular sovereignty. ``The bill is a long one; the assembly has not the
time to devote to it and to give it the scrutiny that a measure of such
importance demands.\textless80\textgreater\textless9d\textgreater{}
Blind approval''would be an injustice to themselves and a greater wrong
and injustice to the people who have a right to demand that their public
servants legislate for the public good." The code ``has been
`assimilated,' as we are informed, `to the character and requirements of
our people,' whatever that may mean,'' but the editor feared the
legislation was the product of ``men who have the welfare of the `dear
people' at their tongue's end always, but never in their
hearts.''\footnote{\emph{Rocky Mountain News}, January 20, 1877.}

The migration of the Field Code was a central event in Anglo-American
legal history, but no historian has traced the extensive borrowings of
the Field text nor recognized the political furor that greeted the code
outside New York.\footnote{Roscoe Pound, ``David Dudley Field: An
  Appraisal,'' and Alison Reppy, ``The Field Codification Concept,'' in
  Alison Reppy, ed., \emph{David Dudley Field: Centenary Essays} (New
  York University School of Law, 1949); Stephen Subrin, ``David Dudley
  Field and the Field Code: A Historical Analysis of an Earlier
  Procedural Vision,'' \emph{Law and History Review} 6 (1988): 311--373;
  Robert G. Bone, ``Mapping the Boundaries of a Dispute: Conceptions of
  Ideal Lawsuit Structure from the Field Code to the Federal Rules,''
  \emph{Columbia Law Review} 89 (1989): 1--118. See also the literature
  on procedure and codification cited below.} Every aspect of a civil
justice system, from the rules granting access to courts, to lawyers, to
remedies (whether damages, injunctions, or seizure of property) was
covered by the code, making its New York-specific rules politically
contentious both inside and outside the Empire State. As the Field Code
migrated around the country, commentators in each jurisdiction raised
the same complaint: how could legislation borrowed from another state
represent the popular will and best interests of \emph{this} state?

Understanding the history of the Field Code requires not only attention
to its political context but also a detailed examination of the
substance of what was borrowed and what was revised in each
jurisdiction. Exploring these borrowings is a daunting task, however.
Procedure codes were long, technical documents, and although each
jurisdiction copied large swaths of text, each also modified the text
along the way, sometimes with a simple \emph{Nevada} for
\emph{California}, sometimes with more foundational changes to civil
remedies. Although Stewart's cut-and-paste code found its way into the
archives, most draft legislation did not, and few codifiers explained in
detail how they produced their texts. Traditional close reading or
textual criticism of some 98,000 distinct sections of law across 20,000
pages comprising 7.7 million words is simply not a feasible research
task for a historian who wishes to track these borrowings.

Yet by turning to the digital analysis of texts, we have resolved this
difficulty and tracked how states borrowed their codes of civil practice
from one another. Within the corpus of legislation, algorithmic analysis
of texts can reverse engineer and visualize what the archive revealed in
figure 1: which texts were borrowed, which modified, and how
extensively.\footnote{We have released two repositories with all the
  code used for this project. Lincoln Mullen, ``textreuse: Detect Text
  Reuse and Document Similarity,'' R package version 0.1.3 (2015--):
  \url{https://github.com/ropensci/textreuse}, includes our
  implementation of LSH and other algorithms suitable for use by other
  scholars. (This package was peer-reviewed by rOpenSci, a collective of
  academic developers who use the R programming language.) A second
  repository contains all of our code specific to the migration of the
  Field Code: \url{https://github.com/lmullen/civil-procedure-codes/}.
  These are the most essential software packages that we used, except
  for those cited elsewhere: R Core Team, ``R: A language and
  environment for statistical computing,'' R Foundation for Statistical
  Computing, Vienna, Austria (2016): \url{https://www.R-project.org/};
  Hadley Wickham and Romain Francois, ``dplyr: A Grammar of Data
  Manipulation,'' R package version 0.4.3 (2016):
  \url{https://CRAN.R-project.org/package=dplyr}; Hadley Wickham and
  Winston Chang. ``ggplot2: An Implementation of the Grammar of
  Graphics,'' R package version 2.1.0 (2016):
  \url{https://CRAN.R-project.org/package=ggplot2}; Hadley Wickham,
  ``stringr: Simple, Consistent Wrappers for Common String Operations,''
  R package version 1.0.0 (2016):
  \url{https://CRAN.R-project.org/package=stringr}; Hadley Wickham,
  \textless80\textgreater\textless9c\textgreater tidyr: Easily Tidy
  Data,\textless80\textgreater\textless9d\textgreater{} R package
  version 0.4.1 (2016): \url{https://CRAN.R-project.org/package=tidyr};
  Gabor Csardi and T. Nepusz, ``The igraph Software Package for Complex
  Network Research,'' \emph{InterJournal, Complex Systems} 1695 (2006):
  \url{http://igraph.org}.} Our method works especially well for legal
texts, for reasons we will explain.

The computational analysis of texts is a method which historians can use
across the discipline to study many topics. We have used text
analysis---specifically, methods to detect text reuse---to show how law
migrates through borrowings, just as scholars on the \emph{Viral Texts}
team have demonstrated how newspaper articles were reprinted in the
nineteenth-century United States.\footnote{Ryan Cordell and David A.
  Smith, \emph{Viral Texts: Mapping Networks of Reprinting in
  19th-Century Newspapers and Magazines}, NULab for Texts Maps and
  Networks, Northeastern University (2012--):
  \url{http://viraltexts.org}.} While we have developed a method that
discovers borrowings of exact words and phrasings, other forms of
digital text analysis can track the diffusion of concepts. Making due
allowance for the specific historical questions and sources at hand,
computational text analysis is broadly applicable to historical problem
that involves the spread of words or ideas. The specific method that we
outline could be extended beyond codes of civil procedure to legal
statutes or treatises. Yet it could also be used, for instance, to track
the spread of hymns in collections of hymnbooks in religious history, or
the reuse of sections of medical textbooks in the history of science.
Since historians by and large work with textual sources and increasingly
with digitized texts, computational text analysis should become a part
of the historian's toolbox. Lara Putnam has recently described the
importance of ``digitization and OCR, which make words above all
available'' for historians to search and read, while observing that
``computational tools can discipline our term-searching if we ask them
to.''\footnote{Lara Putnam, ``The Transnational and the Text-Searchable:
  Digitized Sources and the Shadows They Cast,'' \emph{American
  Historical Review} 121, no. 2 (2016): 399--400.} Historians have been
slow to take up text analysis even as the aid to more traditional
reading that Putnam recommends. Yet as we demonstrate in this article,
such methods can reveal patterns inaccessible to the traditional
historian.\footnote{Only a few historians have used computational text
  analysis, including Cameron Blevins,
  \textless80\textgreater\textless9c\textgreater Space, Nation, and the
  Triumph of Region: A View of the World from
  Houston,\textless80\textgreater\textless9d\textgreater{} \emph{Journal
  of American History} 101, no. 1 (June 1, 2014):
  122\textless80\textgreater\textless93\textgreater47,
  \url{doi:10.1093/jahist/jau184}; Benjamin M. Schmidt, \emph{Sapping
  Attention} (blog): \url{http://sappingattention.blogspot.com/}; Sharon
  Block, ``Doing More with Digitization: An Introduction to Topic
  Modeling of Early American Sources,'' \emph{Common-Place} 6, no. 2
  (January 2006): \url{http://www.common-place.org/vol-06/no-02/tales/};
  Dan Cohen, Frederick Gibbs, Tim Hitchcock, Geoffrey Rockwell, et al.,
  ``Data Mining with Criminal Intent,'' white paper, 31 August 2011,
  \url{http://criminalintent.org}; Robert K. Nelson, ``Mining the
  Dispatch, website, Digital Scholarship Lab, University of Richmond,
  \url{http://dsl.richmond.edu/dispatch/}; Dan Cohen,''Searching for the
  Victorians," 4 October 2010:
  \url{http://www.dancohen.org/2010/10/04/searching-for-the-victorians/};
  Micki Kaufmann, \emph{``Everything on Paper Will Be Used Against Me:''
  Quantifying Kissinger}, digital project (2012--16):
  \url{http://blog.quantifyingkissinger.com/}; E. Thomas Ewing, Samah
  Gad, Bernice L. Hausman, Kathleen Kerr, Bruce Pencek, and Naren
  Ramakrishnan, ``An Epidemiology of Information: Datamining the 1918
  Flu Pandemic,'' project research report (2 April 2014):
  \url{http://vtechworks.lib.vt.edu/bitstream/handle/10919/46991/An\%20Epidemiology\%20of\%20Information\%20Project\%20Research\%20Report_Final.pdf?sequence=1};
  Michelle Moravec, ``\,`Under this name she is fitly described': A
  Digital History of Gender in the History of Woman Suffrage'' (March
  2015): \url{http://womhist.alexanderstreet.com/moravec-full.html}.}

The first contribution of this article is to demonstrate our methods as
applied to a corpus of nineteenth-century civil procedure codes. The
second contribution is to integrate what we learned from the text
analysis with the more conventional approaches of political and cultural
history to explain why the migration of the Field Code mattered. On the
national level the extent of legislative borrowing followed a pattern
American historians have described as a ``Greater Reconstruction'' in
which the former Confederate South and the Far West showed a remarkable
kinship. Scholars have typically described Greater Reconstruction as a
federal development, featuring the creation of national citizenship, a
national economy, and a larger federal apparatus centered in Washington,
D.C. This article shows that Greater Reconstruction had its state-level
dimensions as well. The uniform practice of law and adjudication of
civil remedies was not structured by Washington mandates, however, but
by the anxiety that New York financial capital would follow only New
York civil remedies. At the more local level, our digital computations
can trace modifications within code traditions, for instance, the ways
western and midwestern codifiers altered New York's law to accommodate
hardening conceptions of racial competencies in the civil courts.

\hypertarget{the-origin-and-political-controversies-of-the-field-code}{%
\subsection{1. The origin and political controversies of the Field
Code}\label{the-origin-and-political-controversies-of-the-field-code}}

Until 1848, civil remedies and trial practice in New York were largely
governed by common law traditions loosely categorized as ``practice and
pleadings.''\footnote{When law professors such as New York's David
  Graham Jr.~(a collaborator on the Field Code) began to be appointed to
  university positions, the chair for instruction in legal practice or
  procedure carried this designation of ``practice and pleadings.''
  Today that field is described as ``civil procedure,'' a field that
  grew out of David Dudley Field's codification. The history of practice
  and procedure is a staple of general legal history. See Frederick
  Pollock and Frederick Maitland, \emph{The History of English Law
  Before the Time of Edward I}, 2 vols., 2nd ed.~(Cambridge, 1898);
  Theodore Plucknett, \emph{A Concise History of the Common Law}, 5th
  ed.~(Liberty Fund, 2010); Lawrence M. Friedman, \emph{A History of
  American Law}, 3rd ed.~(Touchstone, 2005). Few book-length works have
  been dedicated to the subject, however. The exceptions are Robert
  Wyness Millar, \emph{Civil Procedure of the Trial Court in Historical
  Perspective} (New York University School of Law 1952); Edward Purcell,
  \emph{Litigation and Inequality: Federal Diversity Jurisdiction in
  Industrial America, 1870--1958} (Oxford, 1992); John H. Langbein et
  al., \emph{History of the Common Law: The Development of
  Anglo-American Legal Institutions} (Aspen, 2009); and to some extent,
  William E. Nelson, \emph{The Americanization of the Common Law: The
  Impact of Legal Change on Massachusetts Society, 1760--1830}, 2nd
  ed.~(University of Georgia Press, 1994).} To understand how to file a
civil claim or to enforce a judgment, a lawyer had to consult ad hoc
statutes from colonial times to the present as well as precedents
reported from cases litigated at common law and chancery. By the 1840s,
enterprising practitioners had collated these materials into a half
dozen marketable treatises, but these remained works of private
opinion---no court was bound to agree with the treatise writers as to
the weight, relevance, or proper interpretation of a legal
statement.\footnote{For the rise of treatises in America generally, see
  G. Edward White, \emph{The Marshall Court and Cultural Change,
  1815--1835} (1988) and A. W. B. Simpson, ``The Rise and Fall of the
  Legal Treatise: Legal Principles and the Forms of Legal Literature,''
  in \emph{Legal Theory and Legal History: Essays on the Common Law}
  (Hambledon Press, 1987), 273--320. The most instructive treatises for
  New York practice in the 1840s were Oliver L. Barbour, \emph{A
  Treatise on the Practice of the Court of Chancery} (1844); Alexander
  M. Burrill, \emph{Treatise on the Practice of the Supreme Court of the
  State of New York}, 2 vols. (1846); David Graham, \emph{A Treatise on
  on the Practice of the Supreme Court of the State of New York}, 3rd
  ed.~(1847); David Graham, \emph{A Treatise on the Organization and
  Jurisdiction of the Courts of Law and Equity in the State of New York}
  (1839); Claudius L. Monell, \emph{A Treatise on the Practice of the
  Supreme Court of the State of New York} (1849); Joseph W. Moulton,
  \emph{The Chancery Practice of the State of New York} (1829), 2 vols.;
  and as a general introduction to the field, the Englishman Henry John
  Stephen's \emph{Treatise on the Principles of Pleading in Civil
  Actions} (2d. ed.~1828).} The common law was accordingly known as
``unwritten law'' despite the proliferation of published texts, because
the common law was not precisely determined until a particular case
demanded resolution.\footnote{See Michael Lobban, \emph{The Common Law
  and English Jurisprudence 1760--1850} (Clarendon Press 1991); Kunal M.
  Parker, \emph{Common Law, History, and Democracy in America,
  1790--1900: Legal Thought before Modernism} (Cambridge: Cambridge
  University Press, 2011); David M. Rabban, \emph{Law's History:
  American Legal Thought and the Transatlantic Turn to History}
  (Cambridge 2013).} Statutes, on the other hand, were ``written law,''
prescribing or reforming the rules even before a case put the precise
question in issue. Within the realm of written law, codes were the
ultimate statutes.\footnote{Or, to use a term from contemporary
  analysis, ``super statutes.'' William N. Eskridge and John Ferejohn,
  \emph{A Republic of Statutes: The New American Constitution} (New
  Haven: Yale University Press, 2010). See also David Lieberman,
  \emph{The Province of Legislation Determined: Legal Theory in
  Eighteenth-century Britain} (Cambridge 1989); Farah Peterson,
  Statutory Interpretation and Judicial Authority, 1776--1860
  (Ph.D.~dissertation, Princeton University, September 2015).}

Codification is, as Lawrence Friedman has written, ``one of the set
pieces of American legal history.'' Law reformers advocated for a
codification of the common law from the earliest days of the Republic
through the Gilded Age, from Massachusetts down to South Carolina.
Efforts ranged from mere compilations of existing statutes in each state
to a full European-style codification meant to be an entirely
comprehensive and systematic statement of the law.\footnote{Friedman,
  \emph{A History of American Law}, 302. See also Charles M. Cook,
  \emph{The American Codification Movement: A Study in Antebellum Legal
  Reform} (1983); Robert W. Gordon, ``The American Codification
  Movement,'' \emph{Vanderbilt Law Review} 36 (1983): 431--458; Maurice
  Eugen Lang, \emph{Codification in the British Empire and America}
  (Lawbook Exchange, 1924).} At its most basic level, codification
proposed that legislative policy ought to be the sole source of law. Law
was to be made by democratically responsible legislators in terse,
unambiguous statements, not discovered through application and analogy
in particular cases by the judges. Debates over codification thus ranged
from the metaphysics of law to political theories of institutional
competency and the separation of powers.\footnote{The most influential
  account has been Morton Horwitz's, which declares that ``the desire to
  separate law and politics has always been a central aspiration of the
  American legal profession'' in order to protect elite interests
  against popular democracy. Horwitz identifies ``orthodox legal
  thought'' and ``orthodox lawyers'' with the elite of the American bar
  who sought to shield the law from political interference, which above
  all meant crusading against legislation and especially codification.
  Morton J. Horwitz, \emph{The Transformation of American Law,
  1780--1850} (Oxford University Press, 1977), 258--59. Recent work has
  challenged Horwitz's account by showing how elite common law lawyers,
  particularly Horwitz's main target James Coolidge Carter, were
  actually political progressives who supported redistributive
  legislation such as the income tax. See, for instance, Rabban,
  \emph{Law's History}, 322--77; Parker, \emph{Common Law, History, and
  Democracy}, 230--41; Lewis A. Grossman, ``James Coolidge Carter and
  Mugwump Jurisprudence,'' \emph{Law \& History Review} 20, no. 3
  (2002): 577--629. These accounts follow Horwitz, however, in focusing
  on the few outspoken opponents of codification, rather than the elite
  lawyers who sponsored the procedure codes. Among the latter group
  could be found some of the most devout theorists of laissez faire
  economics in nineteenth-century America, including David Dudley Field
  and his brother, the Supreme Court Justice Stephen Johnson Field.}
Codification remained a major interest of the American bar across the
nineteenth century. When the intellectual historian Perry Miller
developed a reader surveying The Legal Mind in America, codification was
its central theme, as Miller argued it was the only intellectual topic
that attracted lawyers away from their practices long enough to
debate.\footnote{Perry Miller, ed., \emph{The Legal Mind in America:
  From Independence to the Civil War} (Anchor 1962).}

The importance of codification extended far beyond the United States.
Napoleonic France promulgated a series of codes in the early nineteenth
century, Germans debated the wisdom of codification after
Napoleon\textless80\textgreater\textless99\textgreater s fall, and in
the 1860s the British imposed codes on their colonies in India and
Singapore. Many American lawyers followed the international development
of these codes with interest, viewing codification as the leading edge
of modern legal science.\footnote{See James Q. Whitman, \emph{The Legacy
  of Roman Law in the German Romantic Era: Historical Vision and Legal
  Change} (Princeton, 1990); Gunther A. Weiss, ``The Enchantment of
  Codification in the Common-Law World, \emph{Yale Journal of
  International Law} 25 (2000): 435; Maurice Eugen Lang,
  \emph{Codification in the British Empire and America} (Lawbook
  Exchange, 1924); Jean-Louis Halperin, \emph{The French Civil Code}
  (Austin: University of Texas Press, 2006); Robert B. Holtman,
  \emph{The Napoleonic Revolution} (Baton Rouge: Louisiana State
  University Press, 1981); R. H. Kilbourne, \emph{A History of the
  Louisiana Civil Code} (Paul M. Herbert Law Center, 1987); Brian Young,
  \emph{The Politics of Codification: The Lower Canadian Civil Code of
  1866} (Osgoode Society, 1994); John W. Cairns, \emph{Codification,
  Transplants, and History: Law Reform in Louisiana (1808) and Quebec
  (1866)} (Clark, NJ: Talbot, 2015); Roscoe Pound,''The French Civil
  Code and the Spirit of Nineteenth Century Law," \emph{Boston Law
  Review} 35 (1955): 79. On common theories of codification that
  transcended jurisdictional boundaries, see Csaba Varga,
  \emph{Codification as a Socio-Historical Phenomenon}, 2nd
  ed.~(Budapest: Akadmiai Kiad, 2011 {[}1991{]}); Roger Berkowitz,
  \emph{The Gift of Science: Leibniz and the Modern Legal Tradition}
  (New York: Fordham University Press, 2010).}

In New York, these codification debates came to a head at the 1846
constitutional convention, where ``the conquerors took all,'' as the
ambivalent Jacksonian James Fenimore Cooper complained.\footnote{James
  Fenimore Cooper, \emph{The Ways of the Hour: A Tale} (1850), 84. For
  an analysis of Cooper's philosophy of law and his critique of the New
  York constitution, see Charles Hansford Adams, \emph{``The Guardian of
  the Law'': Authority and Identity in James Fenimore Cooper} (Penn
  State University Press, 1990), 135--48. See also Marvin Meyers,
  \emph{The Jacksonian Persuasion: Politics and Belief} (Stanford,
  1957), 57--100.} Law reformers abolished the court of chancery, made
judges stand for popular election, and required the legislature to
appoint commissioners to codify the law and reform the ``practice and
pleadings'' of the civil courts.\footnote{On the politics and reforms of
  the New York Convention of 1846, see Charles Z. Lincoln, \emph{The
  Constitutional History of New York from the Beginning of the Colonial
  Period to the Year 1905} (Rochester, 1905), 2:10--101; Charles
  McCurdy, \emph{The Anti-Rent Era in New York Law and Politics,
  1839--1865} (University of North Carolina, 2001); Jed Shugerman,
  \emph{The People's Courts: The Rise of Judicial Elections and Judicial
  Power in America} (Harvard University Press, 2012).}

New Yorkers had two models of legislative commissions on which they
could draw for their own law reforms. The French government under
Napoleon had appointed five-member commissions to codify the law of
France. When the New York law reformer William Sampson called for
codification in a widely noted address to the New-York Historical
Society, one of the members of the French commission living in exile in
upstate New York wrote to Sampson. He advised Sampson on the mechanics
of codification: ``Let four or five good heads be united in a
commission, to frame in silence the project of a code. It is not so
difficult a task. It is only to consult together, and to select. Do so
with your best authors as we did with ours, \ldots{} which we simply
converted into articles of our code.'' Tellingly, the French
commissioner took it as granted that the commission's code would
automatically be promulgated as law.\footnote{Count Pierre Franois Ral
  to William Sampson, October 27, 1824, in Sampson's Discourse and
  Correspondence with Various Learned Jurists Upon the History of the
  Law (1826), 191; Maxwell Bloomfield, ``William Sampson and the
  Codifiers: The Roots of American Legal Reform,'' \emph{American
  Journal of Legal History} 11, no 3. (1967): 234--252; Walter J. Walsh,
  ``William Sampson, a Republican Constitution, and the Conundrum of
  Orangeism on American Soil, 1824--1831,'' \emph{Radharc} 5 (2006):
  1--32; William Sampson, \emph{Memoirs}, 2nd ed.~(1817). On
  French-style codifications, see Robert B. Holtman, \emph{The
  Napoleonic Revolution} (Louisiana State, 1981); R. H. Kilbourne,
  \emph{A History of the Louisiana Civil Code} (Paul M. Herbert Law
  Center, 1987); Brian Young, \emph{The Politics of Codification: The
  Lower Canadian Civil Code of 1866} (Osgoode Society, 1994); John W.
  Cairns, \emph{Codification, Transplants, and History: Law Reform in
  Louisiana (1808) and Quebec (1866)} (Talbot, 2015); Roscoe Pound,
  ``The French Civil Code and the Spirit of Nineteenth Century Law,''
  \emph{Boston University Law Review} 35 (1955): 79.}

The other model came from England. Royal commissions had been employed
since before the Revolution of 1688 to advise on a variety of matters.
Although by the mid-nineteenth century royal commissions sometimes
offered model statutes, Parliament maintained exclusive legislative
prerogatives, forcing any commissioner-proposed legislation to pass
through the normal politicking and drafting processes of
Parliament.\footnote{Thomas J. Lockwood, ``A History of Royal
  Commissions,'' \emph{Osgoode Hall Law Journal} 5 (1967): 172; Barbara
  Lauriat, ``\,`The Examination of Everything': Royal Commissions in
  British Legal History,'' \emph{Statute Law Review} 31 (2010): 24;
  Joanna Innes, \emph{Inferior Politics: Social Problems and Social
  Policies in Eighteenth-century Britain} (Oxford, 2009).}

After David Dudley Field and two other lawyers were appointed to the
procedural commission, their reports made clear that they favored the
French model but understood political realities would hold them to the
English model. From 1847 to 1850, the commissioners made five reports to
the legislature knowing they had no power to keep legislators from
amending their code or even defeating it altogether. Each time they
reminded legislators that ``public opinion had issued its mandate in the
most imposing form'' of a constitutional decree. The constitution,
together with the legislative act appointing the commission, ``gave the
commissioners instructions so precise, as to leave them no discretion,
if they had desired it, {[}and{]} promised them therefore in advance, so
long as they obeyed those instructions, the concurrence and co-operation
of all departments of the government.''\footnote{Second Report of the
  Commissioners on Practice and Pleading (New York, 1849), 3--4. See
  also, First Report of the Commissioners on Practice and Pleadings (New
  York, 1848), iii-iv; Third Report of the Commissioners on Practice and
  Pleadings (New York, 1849), 3, Lillian Goldman Law Library Rare Books
  Collection; Final Report of the Commissioners on Practice and
  Pleadings, in Documents of the Assembly of the State of New York, 73d
  sess., vol.~2, no. 16 (New York, 1850), viii.} Although the theory of
codification made it a democratic enterprise, in practice Jacksonians
like Field insisted the democratic legislature ought to defer to the
expertise of the commissioners.

The Field commission sought to blunt criticism by insisting that
political concerns about lawmaking did not apply to mere procedure.
``The system of procedure by which law is administered, differs from the
law itself in this,'' the commissioners explained: ``the latter is a
body of elementary rules founded in the immutable principles of justice,
drawing their origin from the obligations which divine wisdom has
imposed \ldots; while the former consists, in its very nature, but of a
body of prescribed rules, having no source but the will of those by whom
they are laid down.'' Substantive law was universal, natural, grounded
in divine justice, and therefore entitled to respect and protection from
change. But God cared nothing of the ``the mere machinery by which law
is to be administered.'' Thus, the commissioners argued, procedure was
trivial enough for legislative experimentation but complicated enough
that only master practitioners like themselves could run the
experiment.\footnote{Report of the Commissioners on Practice and
  Pleadings, in New York Assembly Documents, 70th sess., vol.~2, no. 202
  (1847), 3--4.}

Yet the code's scope of ``procedure'' included far more than the ``mere
machinery'' of a lawsuit.\footnote{Working under this theory, the Field
  commission defined the content of the modern field of \emph{civil
  procedure}. While western legal systems had long distinguished between
  the law of persons and things on the one hand, and the law of actions
  (the rules of litigation) on the other, in the Anglo-American
  tradition, the categories remained intermixed into the nineteenth
  century. Whether one had a substantive legal right (to property, to
  marry, to an office, etc.) depended upon whether and how one would sue
  for a remedy to vindicate that right. Blackstone's \emph{Commentaries}
  attempted to describe English law in the more European terms of
  persons/things/actions, and Jeremy Bentham offered a more refined
  terminology of ``substantive'' law and ``procedural'' or ``adjective''
  law, but until the Field Code no Anglo-American jurist had specified
  with precision where the line lay between substantive and procedural
  law. See Lobban, \emph{The Common Law and English Jurisprudence},
  127--131, 146--151. As Amalia Kessler notes, \emph{Bouvier's Law
  Dictionary} did not even define \emph{civil procedure} until its 1897
  edition, describing the term as ``rather a modern one.'' Amalia
  Kessler, ``Deciding Against Conciliation: The Nineteenth-Century
  Rejection of a European Transplant and the Rise of a Distinctively
  American Ideal of Adversarial Adjudication,'' \emph{Theoretical
  Inquiries in Law} 10 (2009): 481--482; \emph{Bouvier's Law Dictionary}
  (1897), 2:764. Before 1848, the term was largely restricted to French
  usage, and American remedial law had carried the typical
  designation---as it did both in Graham's treatise and professorial
  title---of ``practice and pleadings,'' the name likewise given to the
  reform commission. When the commission designated its final draft a
  ``Code of Civil Procedure,'' it marked the first American attempt to
  give content to this category.} The final draft of the code, printed
in 1850, spanned nearly 800 pages of 1,885 regulations. The first third
of the code covered constitutional topics, specifying the jurisdiction
of all state courts and the duties of all state officers (and
liabilities for violating those duties). The code deregulated attorney
compensation, introducing novel structures of retainers and contingency
fees.\footnote{Final Report of the Commissioners, 368--378, tit. 10. See
  also Peter Karsten, ``Enabling the Poor to Have Their Day in Court:
  The Sanctioning of Contingency Fee Contracts, A History to 1940,''
  \emph{DePaul Law Review} 47 (1998): 231; Norman Spaulding, ``The
  Luxury of the Law: The Codification Movement and the Right to
  Counsel,'' \emph{Fordham Law Review} 73 (2004): 983; John Leubsdorf,
  ``Toward a History of the American Rule on Attorney Fee Recovery,''
  \emph{Law and Contemporary Problems} 47 (1984): 9.} It created summary
procedures meant to accelerate debt collection while simultaneously
carving out ``homestead'' exemptions from the sheriff's reach.\footnote{On
  debt collection, see Part 4 below. On homestead exemptions, see Final
  Report of the Commissioners, 353--354, 839; James W. Ely, ``Homestead
  Exemption and Southern Legal Culture,'' in Sally Hadden \& Patricia
  Minter eds., \emph{Signposts: New Directions in Southern Legal
  History} (University of Georgia 2013), 289--314; Paul Goodman, ``The
  Emergence of Homestead Exemption in the United States: Accommodation
  and Resistance to the Market Revolution, 1840--1880,'' \emph{Journal
  of American History} 80, no. 2 (1993): 470--498.} The code concluded
by defining who could be an attorney, a juror, and a witness, drawing
racial and gendered distinctions over who could speak in
court.\footnote{Final Report of the Commissioners, 202--203, 506
  (restricting admission as an attorney to male citizens), 110, 251
  (restricting jury service to white male citizens), 714--715, 1708
  (permitting ``all persons, without exceptions'' to be witnesses in
  civil cases).}

Most important, the code defined all the remedies that a civil court
could order---from money damages, to partition of property, to
injunctive decrees and contempts---and made those remedies available in
every lawsuit. In many cases, a legal right was indistinguishable from
the remedy that secured that right: the right to possess a particular
piece of property and the remedy that seized and delivered that property
were, in effect, the same thing. Remedies were thus intimately connected
to substantive law. For that reason, neither the French \emph{code de
procdure civile} (1806) nor Blackstone or Bentham's writings conceived
of remedies as purely procedural.\footnote{On the French procedure code,
  see C. H. van Rhee, \emph{European Traditions in Civil Procedure}
  (Antwerp 2005). Bentham, the leading proponent of codification in
  England, argued that procedure was the one department of the law that
  ought to remain \emph{un}codified. So long as the law of civil and
  criminal obligations and the law of property were sufficiently
  codified, a ``natural procedure'' arising from judicial discretion and
  flexibility would be superior to ``technical'' written rules. Later in
  his career Bentham produced the ``Outlines of a Procedure Code'' as a
  ``provisional'' remedy, but he insisted that a procedure code on its
  own could not be ``invested with the form of law'' without ``reference
  to the codes of law, penal and non-penal, to which it has for its
  object and purpose the giving execution and effect.'' Although it
  spanned nearly 200 pages, Bentham's code favored general moral maxims
  over precise details, for instance: ``On each occasion, have constant
  regard for all the several ends of justice; that is to say, minimize
  the sum, or the balance of evil.'' Jeremy Bentham, Principles of
  Judicial Procedure with the Outline of a Procedure Code, in John
  Bowring, ed., \emph{Works of Jeremy Bentham}, 2nd ed.~(1843
  {[}1839{]}), 1--189, preface and 28, ch.~7 1. See Lobban, \emph{The
  Common Law}, 127--131.} By codifying remedies, Field invited continual
expansion of the category of ``procedure.'' As other states adopted the
procedure code, they sometimes included other fields of law that seemed
obviously ``substantive'' yet had such specific procedures or remedies
that they were placed in a ``code of procedure.'' Such fields included
the law of wills, corporations, and mortgages.\footnote{See Revised
  Statutes of the State of Indiana (1852), 2:245--320 (wills); Public
  Statutes of the State of Minnesota (1859), 643--647, ch.~75
  (mortgages); The Code of Civil Procedure of the State of California
  (1880), 419--420, tit. 6, 657--659, art. 5 (corporations).} And after
all, argued procedural codifiers in Iowa, what did the famed Married
Women's Property Acts offer besides procedural reform? These acts gave
women \emph{standing} to litigate in their own name and seek
\emph{remedies} in claims of property and contract, and they abolished
mandatory rules of \emph{joinder} (of husbands). Thus, one of the most
significant changes to the law of property and domestic relations in the
century went into the state's Code of Civil Procedure.\footnote{Report
  of the Code Commissioners to the Eighth General Assembly of the State
  of Iowa (1859), 296, note to 172 (``The right to sue, follows
  necessarily from the right of property.'') On the significance of the
  Married Women's Property Acts, see, e.g., Hendrik Hartog, \emph{Man
  and Wife in America: A History} (Harvard 2000), 111--113, 187--192,
  290--292.}

Despite Field's arguments that a mere procedure code was democratically
unproblematic, his efforts were not entirely successful in New York. The
commission submitted a draft of its main reforms in 1848, emphasizing
that this first code was ``but a report in part.'' New York's
legislators enacted the partial code with little amendment, some
legislators repeating Field's view that the constitution obligated them
to accept the code.\footnote{First Report of the Commissioners, iv. For
  legislative debates on the Code, see ``Legislative Acts and
  Proceedings,'' \emph{Albany Evening Journal}, Mar.~31, 1848.} But when
the commissioners submitted an extended draft in 1849, the Assembly
judiciary committee balked, directly disputing the commissioners' claims
that procedure was merely the machinery of the law. The ``provisions for
rights and for the mode of pursuing remedies, insensibly run into each
other,'' the committee reported, complicating legal practice
``infinitely more than any machine of human contrivance.'' They
therefore suspected the commissioners' forthcoming code of criminal
procedure would include all of the criminal law as well, ``as they seem
to understand practice and pleadings to include all the law upon a given
subject.'' That being the case, the committee wondered whether they
should ``place in {[}the commissioners{]} a blind and implicit
confidence that shall commit to their discretion the peace and property,
the personal liberty and the lives of those who sent \emph{us} here to
make laws for them?''\footnote{Report of the Committee on the Judiciary
  on the Bill to Continue in Office the Commissioners on Practice and
  Pleadings, in New York Assembly Documents, 72d sess., vol.~3, no. 47
  (New York, 1849), 2, 12--15.}

\begin{figure}
\centering
\includegraphics{field-code-states-map.jpeg}
\caption{This map shows which states adopted codes of civil procedure
based on the New York Field Code. The date shown is the date of the
first enactment of a procedure code; most states subsequently revised
their codes. Note that many southern states and western states came to
adopt the Field Code during the Civil War and Reconstruction. By the end
of the nineteenth century, thirty-one jurisdictions (those displayed on
the map, plus Alaska) had adopted a version of the Field Code. Data
adapted from Charles McGuffey Hepburn, \emph{The Historical Development
of Code Pleading in America and England} (1897).}
\end{figure}

The code would encounter similar difficulties in each jurisdiction that
adopted it. Even the shortest version of the Field Code was
significantly longer than any other state statute before the Progressive
legislation of the twentieth century. Unlike statutory compilations that
sometimes took the name of a ``code'' but made no changes to existing
law, the Field Code opened by abolishing the hallmarks of prior practice
and instituting ``hereafter'' a new form of action with substantial
revisions to basic matters of civil remedies.\footnote{Final Report of
  the Commissioners, 225--226, 554. For examples of ``codes'' that did
  not alter previously enacted statutes, see, for instance, Report,
  Appendix to the Journals of the Senate and Assembly of the State of
  Tennessee (1857), 191 (``The digest presents the law substantially as
  it now exists in the State. I have neither felt at liberty nor deemed
  it advisable to innovate largely upon the established system.''); 1897
  New Mexico Compiled Laws 9 (``The commissioners were given no
  authority to revise.''); 1866 Illinois Compiled Laws v (``We cannot
  change the text, but we can arrange and systematize the entire
  legislation of the state upon any given subject.''); 1849 Wisconsin
  Revised Statutes, ``Advertisement'' (commission ``directed the
  subscriber to arrange the chapters into parts and titles as he thought
  proper, re-arranging the order of the sections or transposing them
  from one chapter to another, whenever it would not alter the meaning
  of the law.'').} In the states where it was imported, there was no
getting around the fact that the code introduced much new law, yet
legislators were unable to read, critique, and amend the code within the
brief period of a legislative session. ``It is folly to undertake to
pass a code in a sixty day session,'' wrote the \emph{Montana Post},
``and the best way would be for the Assembly to select one from a State
or Territory which would come near meeting our wants, and slide it
through with the fewest changes possible.''\footnote{\emph{Montana
  Post}, January 21, 1865.} Sliding the code through eased the problem
of time but exacerbated the problem of local sovereignty. ``To be
governed by a foreign law, especially when that law is not preknown to
the people whose conduct is to be regulated thereby \ldots{} is
something repugnant to the idea of Democratic Republican government,''
complained the \emph{The Miner's Express} in Iowa.\footnote{\emph{The
  Miner's Express} (Dubuque, IA), February 26, 1851.}

How, then, did states and territories achieve a politically acceptable
balance between efficiency and sovereignty, borrowing law for sake of
time but endowing it with popular legitimacy in each locale? Quite apart
from the technicalities of legal practice, the American federation of
civil government into (depending on the year) more than thirty or forty
separate jurisdictions makes it hard to describe a phenomenon that was
truly national despite its state-centered enactments. It requires a
sense of how much law was borrowed in each location and to what degree
innovations were introduced. But precisely because these questions
concern codes---texts that comprehensively and systematically cover a
given subject---they are ideal sources for the techniques of digital
history.

\hypertarget{detecting-borrowings-among-procedure-codes}{%
\subsection{2. Detecting borrowings among procedure
codes}\label{detecting-borrowings-among-procedure-codes}}

To discover how the Field Code migrated to other jurisdictions, we
compiled a corpus of potentially relevant laws, including separately
bound codes of civil procedure as well as codes or statutes appearing
within session laws and statutory compilations from around the Atlantic
world. The corpus comprises 135 statutes from the nineteenth century,
which amounts to 7.7 million words organized into 98,000 regulations. It
includes the initial enactment of every U.S. code of civil procedure, as
well as procedure statutes and re-enacted codes from jurisdictions
reputed to have been legally influential, including French and British
codes. The corpus does not include every nineteenth-century statute of
procedural law. While a comprehensive project may be illuminating in its
own ways, our specific question of how New York legislation influenced
other American jurisdictions permits a more curated corpus.\footnote{For
  full citations to all of the codes that we used, plus links to
  electronic versions at the Hathi Trust, Google Books or other sources
  when available, see Kellen Funk, ``American Civil Procedure: Law on
  the Books'' (2015--16):
  \url{http://kellenfunk.org/civil-procedure/procedure-law/}.}

Curating a corpus to answer a specific question is one of two ways in
which digital history can proceed. To quote Jason Heppler, digital
history--like all historical work---can begin either ``with a corpus
looking for a question, or a question looking for a corpus.''
Computational text analysis in digital history is often conceived of as
beginning with sources, particularly with large datasets such as the
Google Books or Hathi Trust corpora. These large corpora are sometimes
called ``big data''---though it must be emphasized almost never by
digital historians who actually work with them---on which ``distant
reading'' can be practiced. While it is salutary for historians to have
their research questions shaped by the broadest possible contexts, it is
not apparent that digital historians can readily move from these omnibus
corpora to answering the specific research questions that animate
various historical fields. In this article we demonstrate an alternative
approach, which we might unimaginatively label ``medium data.'' The
amount of legislation that governed American civil practice is
impressive, since every state amended and re-enacted procedure statutes
nearly every decade. But while a corpus of procedural legislation
requires some computational sophistication, the techniques are far less
complex than those derived for truly big data. We have gathered a large
but narrowly constrained corpus centered on solving a well-defined
research question.\footnote{Our approach draws on an earlier generation
  of digital history which collected sources, as exemplified in
  \emph{The Valley of the Shadow} project: William G. Thomas III and
  Edward L. Ayers, ``The Differences Slavery Made: A Close Analysis of
  Two American Communities,'' \emph{The American Historical Review} 108,
  no. 5 (December 1, 2003):
  1299\textless80\textgreater\textless93\textgreater1307,
  \url{doi:10.1086/587017}. See also Dan Cohen and Roy Rosenzweig,
  \emph{Digital History: A Guide to Gathering, Preserving, and
  Presenting the Past on the Web} (Philadelphia: University of
  Pennsylvania Press, 2005), ch.~6, digital edition hosted at Roy
  Rosenzweig Center for History and New Media, George Mason University:
  \url{http://chnm.gmu.edu/digitalhistory/}.} This corpus is large
enough that digital historical methods provide results that a scholar
could not obtain through traditional methods, but sufficiently
circumscribed so as to directly address a discipline- and field-specific
question.\footnote{We have preferred to use the term ``digital history''
  when referring to our own work, in part because the term ``digital
  humanities'' has largely come to refer to the work of digital literary
  and digital media scholars, but primarily because we wish to see
  digital scholars make disciplinary, rather than interdisciplinary
  contributions. We are working primarily in the field formerly known as
  humanities computing, but there are other forms of digital history
  such as digital public history or spatial history. On the role of
  disciplines and the importance of field specific argumentation, see,
  Stephen Robertson, ``The Differences between Digital Humanities and
  Digital History,'' 289--307, and Cameron Blevins, ``Digital History's
  Perpetual Future Tense,'' 308--324, both in \emph{Debates in the
  Digital Humanities 2016}, ed.~Matthew K. Gold and Lauren F. Klein
  (Minneapolis: University of Minnesota Press, 2016); William G. Thomas
  III, ``The Promise of the Digital Humanities and the Contested Nature
  of Digital Scholarship,'' in \emph{A New Companion to the Digital
  Humanities}, edited by Susan Schreibman, Ray Siemens, and John
  Unsworth (Malden, MA: Wiley Blackwell, 2016), 524--37. Nevertheless,
  digital text analysis in the humanities has mostly been published by
  literary scholars, including Stephen Ramsay, \emph{Reading Machines:
  Toward an Algorithmic Criticism} (Urbana: University of Illinois
  Press, 2011); Franco Moretti, \emph{Distant Reading} (New York: Verso,
  2013); Matthew L. Jockers, \emph{Macroanalysis: Digital Methods and
  Literary History} (Urbana: University of Illinois Press, 2013); Ted
  Underwood, \emph{Why Literary Periods Mattered: Historical Contrast
  and the Prestige of English Studies} (Stanford: Stanford University
  Press, 2013). Related but less well regarded by humanities scholars is
  work in ``culturomics'': see Jean-Baptiste Michel et al.,
  ``Quantitative Analysis of Culture Using Millions of Digitized
  Books'', \emph{Science} 331, no. 6014 (2011), 176--182;
  \url{doi:10.1126/science.1199644}. On the uselessness of the term
  ``big data,'' see Ted Underwood, ``Against (Talking About) `Big
  Data,'\,'' \emph{The Stone and the Shell}, blog post, 10 May 2013:
  \url{https://tedunderwood.com/2013/05/10/why-it-matters-that-we-dont-know-what-we-mean-by-big-data/}.
  For an overview of digital history projects involving text analysi,
  see the Roberston essay cited above. For an example of curating a
  corpus aimed at research questions, see Ted Underwood, Boris Capitanu,
  Peter Organisciak, Sayan Bhattacharyya, Loretta Auvil, Colleen Fallaw,
  J. Stephen Downie, ``Word Frequencies in English-Language Literature,
  1700--1922,'' dataset, v0.2 (HathiTrust Research Center, 2015)
  \url{doi:10.13012/J8JW8BSJ}.

  Only a few scholars have turned their attention to the computer
  analysis of legal texts for historical purposes, including Paul
  Craven, ``Deteccin automtica y visualizacin de dominios especficos
  similares en documentos: anlisis DWIC y su aplicacin en el Proyecto
  Master \& Servant {[}Automatic Detection and Visualization of
  Domain-Specific Similarities in Documents: DWIC Analysis and its
  Application in the Master \& Servant Project{]},'' published on CD-ROM
  in F. J. A Perez et al., eds., \emph{La Historia en una nueva frontera
  {[}History in a New Frontier{]}} (Digibis: Ediciones de la Universidad
  de Castilla-La Mancha, 1998); Paul Craven and Douglas Hay, ``Computer
  Applications in Comparative Historical Research: The Master \& Servant
  Project at York University, Canada,'' \emph{History and Computing} 7,
  no. 2 (1995); Paul Craven and W. Traves, ``A General-Purpose
  Hierarchical Coding Engine and Its Application to Comparative Analysis
  of Statutes,'' \emph{Literary and Linguistic Computing} 8, no. 1
  (1993): 27--32, \url{doi:10.1093/llc/8.1.27}; Eric C. Nystrom and
  David S. Tanenhaus, ``The Future of Digital Legal History: No Magic,
  No Silver Bullets,'' \emph{American Journal of Legal History} 56, no.
  1 (2016): 150--67, \url{doi:10.1093/ajlh/njv017}; Dan Cohen, Frederick
  Gibbs, Tim Hitchcock, Geoffrey Rockwell, et al., ``Data Mining with
  Criminal Intent,'' white paper, 31 August 2011,
  \url{http://criminalintent.org}.}

Because most codes were public statutes, they were widely printed and
distributed and therefore found their way into libraries digitized by
Google Books. We drew primarily from the Google Books, filling in gaps
from other databases as necessary. We used optical character recognition
software (OCR) to create plain-text versions of the codes, which we
edited lightly, correcting section markers by hand as necessary and
writing a script to fix the most obvious OCR errors.\footnote{In each
  instance we downloaded an entire volume of sessions laws, statutory
  compilations, or single-volume codes of procedure and then cropped out
  irrelevant pages, marginalia and footnoted commentary, leaving only
  the statutory text. After several trials of various implementations of
  Tesseract (open source) and I.R.I.S. (proprietary) OCR programs, we
  determined that Nitro Pro PDF, which relies on I.R.I.S. software,
  offered the best OCR tool for this project. I.R.I.S. provides slightly
  more accurate readings of nineteenth-century typefaces than Tesseract,
  and Nitro Pro's implementation makes words, not characters, the
  fundamental unit of output. The latter feature made cropping between
  marginalia and the statute more reliable. We removed hyphenated line
  breaks and standardized spelling for common terms that evolved over
  the nineteenth century (e.g., \emph{indorsement}).}

The most important step we took in processing the files was to split
each section of the code into its own text file. Codes varied in how
they were organized, but they all divided specific regulations into
\emph{sections} (or, on occasion, \emph{articles}). Not only does the
discursive form of these texts provide a handy organizational scheme for
digital methods, but historically sections were also the way legislators
borrowed their texts. Codifiers took their sources apart by sections,
rearranging here, editing, drafting, and then re-combining there.
Despite the fact that states differed widely on what topics they
included in ``civil procedure,'' sectioning the codes allowed us to
assess similarity even among codes of quite different lengths and
coverage. For instance, we know that California's 1851 code was derived
from New York's 1850 code. (Stephen J. Field, David Dudley's brother,
was the lawyer who imported New York's code into California.\footnote{William
  Wirt Blume, ``Adoption in California of the Field Code of Civil
  Procedure: A Chapter in American Legal History,'' \emph{Hastings Law
  Journal} 17 (1966): 701; Stephen J. Field, \emph{Personal
  Reminiscences of Early Days in California} (1893), 75--78.}) But the
New York code is over 150,000 words long, whereas California's code was
just over 50,000 words long. Those disparate lengths mean that comparing
all of the California code to all of the New York code is less
meaningful than comparing each section in the California code to each
section in the New York code, where matching sections will have a
similar length.

Having divided the texts according to a historically justified pattern,
our next step was to compare each section to every other section and
measure the similarity between them. To continue the New
York-to-California example, consider the following pairs of sections.
The first pair is from the final draft of the New York Field Code. These
sections completely abolished prior practice and began to rebuild the
procedure system from the ground up (figure 3).

\begin{figure}
\centering
\includegraphics{NY1850-text.png}
\caption{Final Report of the Commissioners on Practice and Pleadings
(New York, 1850), 225--25, 554--555.}
\end{figure}

In the theory of Euro-American lawyers, California had no prior practice
to abolish, so the code began more simply (figure 4).

\begin{figure}
\centering
\includegraphics{CA1851-text.png}
\caption{1851 California Laws 51 1--2.}
\end{figure}

The pairs are obviously related to one another, both in terms of their
legal force and in terms of the actual words used.

A common method for measuring the similarity of two documents involves
dividing texts up into tokens of consecutive words (called n-grams) and
calculating a Jaccard similarity score, defined as ratio between the
number of tokens that the two document have in common to the total
number of tokens that appear in both documents. We used five-word tokens
and shingled them, meaning that for the New York sections above, the
first token was ``the distinction between actions at,'' the second token
was ``distinction between actions at law,'' and so on. These tokens each
contain more meaning than a single word, yet because they are shingled
they are robust to changes in the text or noisy OCR. A Jaccard
similarity score will always be in a range between 0 (complete
dissimilarity) and 1 (complete similarity).\footnote{The formal
  definition of the Jaccard similarity for two sets, \(A\) and \(B\), is
  \(J(A, B) = \frac{ | A \cap B | }{ | A \cup B | }\).}

Comparing the section pairs above produces the matrix of similarity
scores in table 1.

\begin{longtable}[]{@{}lcccc@{}}
\caption{A subset of the section-to-section similarity
matrix.}\tabularnewline
\toprule
& NY1850-554 & NY1850-555 & CA1851-001 & CA1851-002\tabularnewline
\midrule
\endfirsthead
\toprule
& NY1850-554 & NY1850-555 & CA1851-001 & CA1851-002\tabularnewline
\midrule
\endhead
NY1850-554 & & 0 & 0.14 & 0\tabularnewline
NY1850-555 & & & 0 & 0.41\tabularnewline
CA1851-001 & & & & 0\tabularnewline
CA1851-002 & & & &\tabularnewline
\bottomrule
\end{longtable}

As we would expect the first sections (New York 554 and California 1)
have a score of 0.14, which indicates that they are similar but have
significant differences, while the second sections (New York 555 to
California 2) have a much higher similarity score of 0.41 since only a
few words were changed. Just as important, when we compare the first
section in New York to the second section in California, we get a score
of 0; the two sections are nothing like each other.

The aim, then, was to create a triangular matrix like the one above, but
with approximately 98,000 rows and 98,000 columns, containing the
similarity scores for each possible pair of sections. While this is easy
enough to conceptualize, such a matrix is actually quite large,
containing about 4.8 billion comparisons. This would take an
unreasonable amount of computation time, and most of these comparisons
would be unnecessary since each section has no relationship to most
other sections. We therefore implemented the minhash/locality sensitive
hashing algorithm to detect pairs of possible matches quickly. Instead
of comparing all tokens to one another, this algorithm samples tokens
from each document to find probable matches, and then Jaccard scores can
be calculated for only those probable matches (that is, many of the
needless calculations that produce scores of zero get cut
out).\footnote{We implemented LSH as described in Jure Leskovec, Anand
  Rajaraman, and Jeff Ullman, \emph{Mining of Massive Datasets}, 2nd
  ed.~(Cambridge University Press, 2014), ch.~3,
  \url{http://www.mmds.org}; the algorithm was first described in Andrei
  Z. Broder, ``On the Resemblance and Containment of Documents,'' in
  \emph{Compression and Complexity of Sequences 1997: Proceedings},
  (IEEE, 1997): 21--29,
  \url{http://gatekeeper.dec.com/ftp/pub/dec/SRC/publications/broder/positano-final-wpnums.pdf}.
  Other digital humanities projects, most notably \emph{Viral Texts},
  have used other means for detecting text reuse. The most prominent of
  these are algorithms for sequence alignment. (Our ``textreuse''
  package for R also implements the Smith-Waterman local sequencing
  algorithm, derived from gene sequencing.) Yet the older and simpler
  LSH algorithm sufficed for our purposes because legal sources are
  easily divided into discrete sections which can be treated as
  independent documents. For other approaches, see David Bamman and
  Gregory Crane, ``Discovering Multilingual Text Reuse in Literary
  Texts,'' white Paper, Perseus Digital Library (2009):
  \url{http://www.perseus.tufts.edu/publications/2009-Bamman.pdf};
  Timothy Allen, Charles Cooney, Stphane Douard, et al., ``Plundering
  Philosophers: Identifying Sources of the Encyclopdie,'' \emph{Journal
  of the Association for History and Computing} 13, no. 1 (2010):
  \url{http://hdl.handle.net/2027/spo.3310410.0013.107}; Glenn Roe,
  Russell Horton, and Mark Olsen, ``Something Borrowed: Sequence
  Alignment and the Identification of Similar Passages in Large Text
  Collections,'' \emph{Digital Studies / Le Champ numrique} 2, no. 1
  (2010):
  \url{http://www.digitalstudies.org/ojs/index.php/digital_studies/article/view/190/235};
  David A. Smith, Ryan Cordell, and Elizabeth Maddock Dillon,
  ``Infectious Texts: Modeling Text Reuse in Nineteenth-Century
  Newspapers,'' in \emph{2013 IEEE International Conference on Big
  Data}, 2013, 86--94, \url{doi:10.1109/BigData.2013.6691675}; David A.
  Smith, Ryan Cordell, Elizabeth Maddock Dillon, et al., ``Detecting and
  Modeling Local Text Reuse,'' \emph{Proceedings of IEEE/ACM Joint
  Conference on Digital Libraries} (IEEE Computer Society Press, 2014);
  Christopher Forstall, Neil Coffee, Thomas Buck, Katherine Roache, and
  Sarah Jacobson, ``Modeling the scholars: Detecting Intertextuality
  through Enhanced Word-level N-gram Matching'' \emph{Digital
  Scholarship in the Humanities} 30, no. 4 (2015): 503--515; Douglas
  Ernest Duhaime, ``Textual Reuse in the Eighteenth Century: Mining
  Eliza Haywood's Quotations,'' \emph{Digital Humanities Quarterly} 10,
  no. 1 (2016):
  \url{http://www.digitalhumanities.org/dhq/vol/10/1/000229/000229.html}.}

The result was a matrix of similarity scores, with over 45,000 genuine
matches. For each section in the corpus, after making some adjustments
to remove anachronisms and spurious matches, we were able to identify
the section from which it was most likely borrowed.\footnote{We filtered
  this matrix based on what we knew about the process of borrowing. We
  removed any match below a threshold that we determined by checking a
  sample of matches. Because Jaccard similarity scores are symmetric, we
  also removed anachronistic matches. For instance, a code from 1851
  obviously did not borrow from a code from 1877. Furthermore, in chains
  of borrowing (e.g., NY1850 to CA1851 to CA1868 to CA1872 to MT1895)
  the latest section might have a high similarity to all of its
  ancestors, but it was in fact borrowed only from the most recent
  parent. We therefore filtered the similarity matrix to remove matches
  within the same code, anachronistic matches, and spurious matches
  beneath a certain threshold. Then if a section had multiple matches,
  we kept the match from the chronologically closest code, giving
  preference to codes from the same state, unless there was a
  substantially closer match from a different code.} In other words, we
had traced the work of the commissioners' scissors and paste-pots
through the course of their codes.

\hypertarget{patterns-of-borrowing-among-field-code-jurisdictions}{%
\subsection{3. Patterns of borrowing among Field Code
jurisdictions}\label{patterns-of-borrowing-among-field-code-jurisdictions}}

The computational evidence that we assembled revealed patterns in how
law migrated at several different scales of analysis.\footnote{Attention
  to big and small scales is described in Shawn Graham, Ian Milligan,
  and Scott Weingart, \emph{Exploring Big Historical Data: The
  Historian\textless80\textgreater\textless99\textgreater s Macroscope}
  (Imperial College Press, 2015).} We used the similarity matrix as the
input to three different digital history techniques: network analysis,
visualizations, and clustering.

At the broadest scale of analysis, we aggregated the section-to-section
borrowings into a summary of how many sections each code borrowed from
each other code. We therefore can show the connections from one code to
another. The resulting network graph reveals the genealogy of civil
procedure in the United States.

\begin{figure}
\centering
\includegraphics{Funk-Mullen.Spine-of-American-Law_files/figure-latex/code-to-code-network-1.pdf}
\caption{The structure of borrowings among nineteenth-century codes of
civil procedure. Note that several versions of New York's Field Code
were at the center of the network, while other states such as California
and Ohio became centers of regional variations on the Field Code. States
that adopted any of the variations on the Field Code became part of a
network centered on New York capital.}
\end{figure}

The New York Field Codes, especially the finished draft of 1850, were
central to the entire network.\footnote{That New York codes are central
  is obvious from the visualization, but we also confirmed this by
  formal measures of centrality used in network analysis. A network is
  simply a list of edges (in our case the number of sections borrowed)
  between nodes (in our case, the codes). Because even our efforts at
  determining the best match for each section sometimes attributed a
  section to an incorrect code, we pruned the edges of the graph so that
  each code was connected to another code only if it borrowed at least
  fifty sections or twenty percent of its sections.

  Within New York there was a definite chronological progression from
  the 1848, 1849, 1850, 1851, and 1853 versions of the code, but the
  development was not chronologically linear. The state legislature
  enacted the 1848, 1849, and 1851 codes, and these show strong
  similarities in their relationships. The 1850 and 1853 versions were
  David Dudley Field's ideal drafts of the code which were never
  enacted. They were, however, printed with wide margins, quality
  typesetting, and---in the 1850 draft---extensive explanatory notes,
  all with an eye towards other jurisdictions copying them as a model.
  Those two codes show stronger similarity to one another than to the
  enacted drafts.} New York gave rise to different regional traditions
within the procedural network. Variations in the Field drafts meant that
different states could borrow different versions of the Field Code.
Field\textless80\textgreater\textless99\textgreater s 1850 draft---never
actually enacted in New York---was the primary progenitor of several
families of codes in California, Kentucky, Iowa, and Ohio, each of which
in turn became major contributors to the law of neighboring states. The
1851 New York code---a small revision to the original 1848 code---became
the progenitor of codes for Wisconsin, Florida, North Carolina, and
South Carolina. While Field considered the 1850 version to be the
definitive, ideal version of the code, all of the New York codes from
1848 to 1853 became models for other jurisdictions. In many cases, the
commissions likely used whatever version of the code they had at hand.
The Field Code was not a single volume on the shelf, but a series of
drafts, any of which might be more accessible in different regions and
in different years.

Even later New York codes can be considered a separate family. In 1876 a
New York commission produced a new code attempting to consolidate all
the case law and statutory amendments subsequent to the 1851 Field Code.
David Dudley Field was upset by the changes introduced in this revision.
A count by a ``friend'' of Field's found that only three sentences of
the Field Code had carried over word-for-word into the latest edition.
With respect to Mr.~Field or his ``friend,'' we found that the
connection between the codes somewhat stronger than he thought, although
his conclusion that, textually, the 1876 code did ``not appear to be the
same thing as before,'' remains sound.\footnote{David Dudley Field, The
  Latest Edition of the New York Code of Civil Procedure (1878), 21.}

Finally, our corpus included a number of statutes which stood outside
the Field Code tradition, such as Virginia and West Virginia
regulations, statutes from Massachusetts and Maine, and southern codes
from Georgia to Louisiana. These statutes show that the dominance of the
Field Code was not total, and a number of older jurisdictions remained
outside of its ambit.\footnote{Non-Field jurisdictions occasionally
  exhibited a borrowing relationship within a state or across two
  states. In in a few unusual instances they contributed to codes which
  were derived from the Field Code. For example, Alabama's 1852 Code
  provided a few sections to Tennessee's 1858 code, and some states like
  Wisconsin copied, along with the Field Code, large passages of
  pre-code legislation from earlier in the state's history.} But nearly
every jurisdiction established or reconstructed after 1850 became a part
of the Field Code network, and no other tradition achieved anywhere near
the same coherence across state lines.

In addition to the overview of the relationship between codes, we can
also see more detail by visualizing the pattern of borrowings within
each code. To illustrate this, we will follow one branch of the Field
Code network, beginning with the family started by California's 1850 and
1851 codes.

California's 1850 code, enacted in the period when California was
entering the Union as a state, was borrowed almost entirely from New
York's 1849 Field Code. The compiler Elisha Crosby did lift one portion
from the mixed civilian/common law code of Louisiana, the rules for
ordering a new trial to revisit an earlier
jury\textless80\textgreater\textless99\textgreater s verdict. New trials
were not provided for in the New York Code until the finished draft in
1850. Most of the sections that were not borrowed, as with many of the
codes, have to do with parts that describe the system of courts or
provide sample forms of pleading or
sheriff\textless80\textgreater\textless99\textgreater s writs that were
peculiar to each state.

\begin{center}\includegraphics{Funk-Mullen.Spine-of-American-Law_files/figure-latex/CA1850-borrowings-1} \end{center}

When Stephen Field revised
California\textless80\textgreater\textless99\textgreater s code in 1851,
he largely redrafted it from the updated code his brother David Dudley
had completed for New York in 1850. This includes the portion of the
code on new trials previously borrowed from Louisiana.\footnote{Final
  Report (New York, 1850), 804--809, compared to 1851 California Laws
  260, 439--441.} The remainder of the code was borrowed from the 1850
California code. (Many of the non-matching sections are tables of
contents.) Thus California based the majority of its law of civil
remedies entirely on New York's code not once, but twice. California
made few to no innovations to the Field Code beyond a rearrangement of
its provisions and their application to the new
state\textless80\textgreater\textless99\textgreater s particular system
of courts.

\begin{center}\includegraphics{Funk-Mullen.Spine-of-American-Law_files/figure-latex/CA1851-borrowings-1} \end{center}

The pattern of borrowings in the Washington 1855 code was a rather
different case. The Washington code was definitely in the lineage of the
1851 California code, since it borrowed sections from both California
directly as well as from Oregon (which was also derived from
California). Indiana's 1852 code and Oregon's 1854 code provide the
majority of the borrowings. The contiguous bands of borrowings
correspond to regulations on judgement borrowed from Oregon and
enforcement provisions borrowed from Indiana. This pattern likely came
about because one of the Washington code commissioners, Edward Lander,
was an Indiana appellate judge from 1850 to 1853, while another
commissioner, William Strong, was a justice of the Oregon Supreme Court
in the same years. While working on the Washington code, they must have
each used the law (and the law books) that they knew best. As a second
generation variation on a regional code, the Washington code drew from a
variety of sources, even though all these sources basically agreed on
the substance of the law.

\begin{center}\includegraphics{Funk-Mullen.Spine-of-American-Law_files/figure-latex/WA1855-borrowings-1} \end{center}

Finally, we can examine one of the outermost leaves on the family tree
of the Field Code in
Washington\textless80\textgreater\textless99\textgreater s revised code
of 1873. The bulk of this code was taken from the early Washington code
with only small amendments. The main exception was lengthy set of
sections on probate drawn from California's 1872 code. Like many of the
last generation codes, the text of the procedure code had stabilized as
a local manifestation of a regional tradition. The code was still
genuinely a Field Code, with a great deal of similarity to the original
New York Field Codes, but its specific form depended on the many edits
and rearrangements that code commissioners from several states had made
to the text.

\begin{center}\includegraphics{Funk-Mullen.Spine-of-American-Law_files/figure-latex/WA1873-borrowings-1} \end{center}

So far we have retained the context of the surrounding sections within a
particular code. But since our fundamental unit of comparison is section
to section, we can use a technique called clustering to group sections
based on their similarity to one another, regardless of which code they
come from. There are innumerable clustering algorithms, but we used the
affinity propagation clustering algorithm because its assumptions
aligned with the characteristics of our problem. That algorithm finds an
``exemplar'' item which is most characteristic of the other items in the
cluster. That assumption fits nicely with borrowings from the Field
Code, where a single section (likely from a Field Code) had many
borrowings, but where there could also be innovative sections from other
states that might be more influential.\footnote{Brendan J. Frey and
  Delbert Dueck, ``Clustering by Passing Messages Between Data Points,''
  \emph{Science} 315 (2007): 972--976, doi: 10.1126/science.1136800;
  Ulrich Bodenhofer, Andreas Kothmeier, and Sepp Hochreiter,
  ``APCluster: An R package for Affinity Propagation Clustering,''
  \emph{Bioinformatics} 27 (2011): 2463--2464,
  \url{doi:10.1093/bioinformatics/btr406}. Even though the affinity
  propagation algorithm did not fully converge with our dataset, it did
  an adequate job clustering the documents. Because there was an
  exemplar section for each cluster, we were able to merge clusters
  whose exemplars had a high Jaccard similarity score.}

The result was a set of approximately 2,900 clusters which contained at
least five sections, though this probably overstates the number of
innovative, \emph{ur}-sections in the corpus. The biggest cluster, which
concerned the use of affidavits in pleading, contained 103 sections.
Within each cluster, we organized the sections chronologically. We were
thus able to see the development of the law from jurisdiction to
jurisdiction over time. This method provides historians with a way of
noticing small changes in the wording and substance of the law. Most
discussions of algorithmic reading have focused on ``distant reading,''
or have balanced the claims of distant reading by using it as a means to
enable close reading. This method of clustering, however, is a kind of
algorithmic close reading. By deforming the texts---taking them out of
the context of the codes and putting them into the context of their
particular variations---we are able to pay attention to those
variations.\footnote{Lisa Samuels and Jerome McGann, ``Deformance and
  Interpretation'' \emph{New Literary History} 30, no. 1 (1999): 25--56;
  Mark Sample, ``Notes toward a Deformed Humanities,'' blog post, 2 May
  2012:
  \url{http://www.samplereality.com/2012/05/02/notes-towards-a-deformed-humanities/};
  Ramsay, \emph{Reading Machines}, 32--57.}

Take provisions regulating witness testimony as an example. At common
law, parties and interested witnesses were not permitted to testify in
their own causes. Field's Code reversed this rule, expanding witness
competency as widely as possible: any person ``having organs of sense''
was to be admitted as a witness in New York, with only the insane and
very young children possibly exempted. As the code migrated West,
however, legislators added racial exclusions to Field's list. The
cluster of sections in the appendix documents how California's codifiers
grafted earlier prohibitions from midwestern states into Field's Code.
Many other codes then evidenced a remarkable uniformity with
California's text (which later changed only to add ``Mongolian'' to the
list of races). Iowa's Code had a more minor influence, and Wyoming
developed the only truly original section which made it explicit that
the exclusion was based on the racist assumption that non-white peoples
were infantilized, a connection only implicit in other jurisdictions.

Uniformity in the law, as shown through these clusters, is just as
instructive as variation. The most significant clusters that we
investigated related to the collection of debts. These were clusters
which went against the typical pattern we observed. While most clusters
exhibited regional variation as they grew more distant from the Field
Code, clusters having to do with creditors' remedies were almost
completely uniform across the American West and South. No single section
of the code announced its preference for creditors' rights; rather, the
acceleration of creditors' remedies resulted from the combination of
several sections. In New York's original enacted code from 1848, 107
required a defendant to answer the complaint within twenty days, instead
of at the next court session (which in some cases could have been as far
as three months away); 202 provided for default judgment as a matter of
course, issued by a clerk without a judicial order if no adequate answer
was received within the twenty days; 128--133 abolished fictitious
pleadings and required answers to state true facts verified by a
defendant's oath, all so that no trial would delay the enforcement of
uncontestable obligations; finally, the code abolished a traditional
thirty-day waiting period between issue of judgment and commencement of
execution. These provisions dealt with what merchants and capitalists
perceived as an abuse of the common law system, where defendants in
cases of debt could stretch out enforcement of debt collection for as
long as two years. The Field Code's summary judgment brought down the
time of debt collection to a matter of weeks. The code thus traded the
rhythms of agriculture for the rhythms of merchant finance.\footnote{First
  Report (New York, 1848), 197.}

Clustering each of these sections reveals that western states along with
the former Confederate states of South Carolina, North Carolina, and
Florida copied each provision almost exactly. Midwestern and Upper South
states that had already developed and maintained commercial ties to
Chicago and New Orleans by 1850 varied the New York rules, sometimes by
requiring answers only in term time, or permitting only a judge to
decree default judgment rather than a clerk, in either case effectively
stretching out enforcement and making a formal trial more likely. But in
the Reconstruction South, and in the West over the same period,
regardless of whether a jurisdiction abolished chancery or not,
regardless of the racial exclusions it may have placed on witness
testimony, the provisions on debt collection remained unchanged. When it
came to creditors' remedies, the law of New York became the law of the
land.

\hypertarget{procedure-codes-and-american-capitalism}{%
\subsection{4. Procedure codes and American
capitalism}\label{procedure-codes-and-american-capitalism}}

The near uniformity of creditors' remedies in Field Code states, as
demonstrated by clustering, points out the close link between the rise
of modern American procedure and modern American capitalism. Westerners
and southerners frequently commented on the seeming imperialism of the
New York code and its connections to New York capital, but one must turn
to the technical debates over procedure to find these anxieties. Twelve
of the states and territories that copied the Field Code most closely
did so during the Civil War and Reconstruction era---four states in the
former Confederacy and eight jurisdictions in the Far West.\footnote{Those
  jurisdictions were Nevada (1861), Dakota Territory---which retained
  the Code when split into North and South (1862), Idaho (1864), Arizona
  (1864), Montana (1865), Arkansas (1868), North Carolina (1868),
  Wyoming (1869), Florida (1870), South Carolina (1870), Utah (1870),
  and Colorado (1877).} As with other areas of postbellum history, it
turns out one may learn a lot by holding the postbellum American South
and American West together.

In the last decade, scholars of Reconstruction have broadened the scope
of their study to include both the South and the West as two sites in
one ``Greater Reconstruction.'' These studies have illustrated the ways
that military conquest, rapid industrialization, and the resettlement
and education of ethnic minorities developed similarly in each region,
guided by and political elites in Washington.\footnote{Elliott West,
  ``Reconstructing Race,'' \emph{Western Historical Quarterly} 34
  (2003): 6. See also Heather Cox Richardson, \emph{West from
  Appomattox: The Reconstruction of America after the Civil War} (Yale
  University Press, 2007); Sven Beckert, \emph{Monied Metropolis: New
  York City and the Consolidation of the American Bourgeoisie,
  1850--1896} (Harvard, 2003); Mark Wahlgren Summers, \emph{The Ordeal
  of the Reunion: A New History of Reconstruction} (University of North
  Carolina Press, 2014). The major application of the Greater
  Reconstruction idea to legal history has been Sarah Barringer Gordon,
  \emph{The Mormon Question: Polygamy and Constitutional Conflict in
  Nineteenth-Century America} (University of North Carolina Press,
  2003).} In tracing the legal aspects of this Greater Reconstruction,
scholars have focused almost entirely on the expansion of federal power
or constitutional rights of citizenship and civil equality.\footnote{See,
  for instance, Rogers Smith, \emph{Civic Ideals: Conflicting Visions of
  Citizenship in U.S. History} (Yale, 1999); Meg Jacobs, William J.
  Novak, and Julian E. Zelizer, eds., \emph{The Democratic Experiment:
  New Directions in American Political History} (Princeton, 2003);
  Edward Purcell, \emph{Litigation and Inequality: Federal Diversity
  Jurisdiction in Industrial America, 1870--1958} (Oxford, 1992).} While
the 1860s and 1870s were of course a transformative period in the
history of civil rights and the creation of a national state, they were
also the decades in which many local legal institutions and practices
were transformed not by federal power but by state codification. Naomi
Lamoreaux and John Joseph Wallis have recently argued that in the
creation of a modern American economy, ``the federal government played
\emph{no} role in this process until the Civil War, and even then it
played only a bit part.'' The history of the Field Code's migration
helps to substantiate this claim. While Lamoreaux and Wallis focus on
the development of banking, transportation, and incorporation at the
state level, it was the Procedure Code that structured civil remedies to
protect these institutions. And procedure codes were creatures of the
states.\footnote{Naomi Lamoreaux and John Joseph Wallis, ``States, Not
  Nation: The Sources of Political and Economic Development in the Early
  United States,'' paper presented at the Economic History Workshop,
  Harvard University, March 4, 2016.}

The states that adopted the Field Code had other options available to
them. Southern states with a civilian code tradition such as Louisiana
and Alabama offered alternative ways to reform common law
practice.\footnote{For instance, although Tennessee in 1858 adapted
  nearly 225 sections of its code from Field codes, the state also
  incorporated nearly 50 sections of the 1852 Code of Alabama, one of
  the largest borrowings of southern legislation within the corpus we
  collected.} Illinois, on the other hand, long retained the common
law---in later decades lawyers called it ``the Yellowstone Park of
common law pleading.'' When Colorado was a territory it imported
Illinois common law via statute, a full seventeen years before its
legislature considered the Field Code.\footnote{See Charles E. Clark,
  ``The New Illinois Civil Practice Act,'' \emph{University of Chicago
  Law Review} 1 (1933): 209. Nevertheless, even Illinois had substantial
  legislation organizing the courts and prescribing certain common law
  processes for obtaining civil remedies, and Colorado adapted the bulk
  of this legislation when it organized as a territory, a full seventeen
  years before its legislature considered adopting the Field Code.}

But what codifiers saw when they looked at New York, more so than at
Louisiana, Alabama, or Illinois, was the Empire State of commercial
capital. The fears, demands, and desires of a personified Capital
continually wielded promises---and threats---in the debates over
procedural codification. The early Mormon settlers of Utah persistently
avoided the mining frenzy as well as codification. By 1870, however, the
territory's governor directed the legislature's attention to the recent
Code of Nevada, a code ``for a people whose interests in many respects
are similar to our own.'' Of course, standing behind the Nevada Code was
``the State of New York---a State which is an empire in itself and whose
commercial transactions are far greater than those of any other State in
the Union.'' By copying its code, Utah too could be ``rewarded by equal
advantages.''\footnote{Journal of the Assembly of the Territory of Utah
  (1870), 15. See also Leonard J. Arrington, \emph{Great Basin Kingdom:
  An Economic History of the Latter-Day Saints, 1830--1900}, new edition
  (University of Illinois Press, 2004).} Code proponents in Colorado
similarly pointed to the fact that the code had been
\textless80\textgreater\textless9c\textgreater adopted twenty-nine years
ago by the Empire state of the
Union,\textless80\textgreater\textless9d\textgreater{} and they too
hoped that the code of the
nation\textless80\textgreater\textless99\textgreater s commercial empire
brought wealth in its wake.\footnote{``The Code Again,'' \emph{Pueblo
  Daily Chieftain}, February 25, 1877.}

When a Colorado legislator scoffed at the idea that capitalists could
possibly care about the difference between old common law and modern
code remedies, his adversaries rebuked him. ``Mr.~Hamill replied that he
knew of one California company of capitalists who were deterred from
investing in mining property here wholly on account of the practice of
the courts in mining cases. If we had had this code years ago, Colorado
would now have a larger amount of California capital in her
mines.''\footnote{``The Legislature: The Senate Devotes Another Day to
  the Code,'' \emph{Denver Daily Tribune}, February 17, 1877.} Codifiers
argued that, in attracting capital, procedure was at least as important
as the substantive rules of property and contract, because procedure
secured the remedies that actually protected investments. ``Men of
capital and enterprise will not make investments and devote their time
and energies to those works of internal improvement,'' Nebraska's
governor reasoned, ``unless ample protection is afforded them, by legal
enactment, for the capital invested and labor employed.'' He therefore
urged swift passage of the Field Code.\footnote{Governor's Message, in
  Journal of the House of Assembly of the Territory of Nebraska (1857),
  12.}

Receiving innumerable letters complaining that under the code, ``no one
will be benefited, except perhaps some Northern Capitalists,'' a North
Carolina commissioner undertook an anonymous defense of the new code in
the \emph{Weekly Standard}. He encouraged the bar to accommodate
themselves to change, for ``the New York system \ldots{} bids fair to
become national.'' Purporting to give an overview of the code, the
articles were almost entirely about credit. ``How can we create credit?
By punctuality,'' the commissioner wrote. ``And how create punctuality?
by law, and by law alone. Let the law enforce punctuality; let the
people of North Carolina learn that the great law of business is, that
`time is of the essence of the contract.'\,'' Under the more certain and
speedy remedies of the code, ``we may expect that \ldots{} even that the
vaults of the banks of New-York \ldots{} will be open to our
industry.''\footnote{William A. Jenkins to William Blount Rodman,
  January 14, 1868, Rodman Papers. East Carolina University Library,
  Special Collections. Rodman's explication of the Code appeared in
  three sequentially numbered articles in the \emph{Standard} on August
  14, 15, and 16, 1868, under the title ``The Code of Civil Procedure.''
  Rodman disclosed his authorship in private correspondence with
  Barringer. See Barringer to Rodman, August 21, 1868, Rodman Papers.}

As in postbellum North Carolina, establishing a flow of credit through
the remedial system became a leading priority of western lawyers. While
the new western history has shed significant light on neglected topics
of Indian dispossession and environmental management, it has often done
so by leaving out of view matters of political economy, a staple of the
old western history. As one work in that older tradition argued, ``Debt
collection was the central part of law practice for the {[}western{]}
bar and remained a key part of private practice throughout the
century.''\footnote{Gordon Morris Bakken, \emph{Practicing Law in
  Frontier California} (Nebraska 1991), 51--54. The new western history
  ushered in by Patricia Nelson Limerick, \emph{The Legacy of Conquest:
  The Unbroken Past of the American West} (Norton, 1987) and Richard
  White, \emph{``It's Your Misfortune and None of My Own'': A History of
  the American West} (Oklahoma, 1991) is now returning to issues of
  political economy. See Patricia Nelson Limerick, \emph{A Ditch in
  Time: The City, the West, and Water} (Fulcrum, 2012); Richard White,
  \emph{Railroaded: The Transcontinentals and the Making of Modern
  America} (Norton, 2011).} On that understanding, one western lawyer
succinctly summarized the difference between the code and the common law
as ``whether a merchant had better try to collect a \$500 note or burn
it up.'' Tiring of all the focus on creditors' remedies, one legislator
observed that he ``never knew one of these professionals who undertook
to write up the beauties of the New York code, \ldots{} that he did not
also break out somewhere with `take for instance the case of an action
on a promissory note,' as though the collection of notes was about all
there could be any law needed for.''\footnote{``The Code,'' \emph{Denver
  Daily Tribune}, January 10, 1877; ``The Code,'' \emph{Denver Daily
  Times}, January 27, 1877.}

The creditors' remedies in the code gave the codifiers their leading
argument against criticisms rooted in the ideology of popular
sovereignty. ``There is no doubt but the people are in favor of anything
that promises to hurry up \ldots{} Justice, and they will go for the old
code,'' one Colorado newspaper announced.\footnote{``A Code of Civil
  Procedure,'' \emph{Denver Daily Times}, January 12, 1877.} New York's
``code practice is the best in excellence,'' stated another, ``and when
I say \emph{best} I do not mean best for lawyers only, but best for the
people---the commonwealth.''\footnote{``The Code,'' \emph{Denver Daily
  Tribune}, January 31, 1877.} If the people favored economic progress,
certainty of remedy, and efficiency in proceedings, then they favored
the New York code, no matter whether they understood or cared about
technical rules of pleading and remedies. Thus, in their arguments the
codifiers imagined themselves the champions of popular sovereignty, for
it was they who accomplished what the people actually desired.

Thus by the end of Reconstruction, New York's domestic empire of capital
and creditors' remedies bore a remarkable resemblance to the
international empire administered by England. Both jurisdictions, while
reforming the practice of law, remained ambivalent about codification
within their own borders but encouraged it among their economic
dependents. The English commissioned codes for India and Singapore,
while Field\textless80\textgreater\textless99\textgreater s additional
codes covering New York\textless80\textgreater\textless99\textgreater s
civil and penal law---ignored in his home state---were adopted in
California and other western jurisdictions.\footnote{See Gunther A.
  Weiss, ``The Enchantment of Codification in the Common-Law World,''
  \emph{Yale Journal of International Law} 25 (2000): 435. For a
  thorough study of the ideology of codification in India, see Robert A.
  Yelle, \emph{The Language of Disenchantment: Protestant Literalism and
  Colonial Discourse in British India} (Oxford: Oxford University Press,
  2012).} In both England and New York, leading arguments against
codification carried a civilizational logic of empire: advanced
metropoles could not codify their law, for to do so would be to freeze
the progress of legal science. What appeared to some to be a hopeless
mass of confusion was to common law defenders the sign of true legal
sophistication. Science was, after all, complex.\footnote{See, for
  instance, James C. Carter's classic defense of the common law against
  codification, ``The Ideal and the Actual in the Law,'' Address to the
  American Bar Association, August 21, 1890, at 28 (``the legislature
  should never attempt to perform the function of the judge, that of
  simply ascertaining and declaring existing customs. This is the work
  of experts who can qualify themselves only by the devotion of their
  lives.'').} The later editions of the New York Procedure Code came in
for censure precisely for trying to capture all the sophistication of
the New York legal system within an unwieldy 3,300 rules.\footnote{See,
  for instance, ``Note,'' \emph{Albany Law Journal} 29 (1884): 141--42;
  Millar, \emph{Civil Procedure of the Trial Court}, 55--56.}
Codification, however, could help developing societies along law's
frontier take a progressive leap forward. As India's chief codifier
Thomas Macaulay explained, codification ``cannot be well performed in an
age of barbarism,'' but also ``cannot without great difficulty be
performed in an age of freedom.'' As India balanced between the two,
however, ``it is the work which especially belongs to a government like
that of India---to an enlightened and paternal despotism.''\footnote{19
  \emph{Hansard Parliamentary Debates} 531.}

In America, Macaulay's tool of enlightened despotism spread with the
anxiety that capital from the nation's economic center would remain
scarce without a code of remedies that, if not in fact the law of New
York, was at least prescribed by New York lawyers and their corporate
clients. In the two most populous and commercially advanced western
states, Texas and Illinois, the need for New York capital failed to move
state legislators to adopt the code at the expense of popular
sovereignty (despite concerted efforts in both jurisdictions).\footnote{Texas
  commissioned the preparation of a code of civil procedure in 1855, and
  the legislature scheduled an extra session to consider it but
  ultimately never passed the law. 21 \emph{Texas Reports} (Hartley) xi
  (1882); \emph{Texas State Times}, December 15, 1855. Reformers in the
  1869 convention in Illinois attempted to pass a provision similar to
  the one in New York's 1846 constitution, which would have required the
  legislature to appoint a commission to revise practice and pleadings
  along the lines of the Field Code. See \emph{Debates and Proceedings
  of the Constitutional Convention of the State of Illinois} (1870),
  1496--1498.} Lacking the self-sufficiency of those two jurisdictions,
the other states of Greater Reconstruction in America adopted a foreign
code, but lawyers, legislators, and their supporters claimed the
endorsement of popular sovereignty in doing so. Even in North Carolina,
whose Democratic newspapers daily called for the repeal of ``this child
of the carpet baggers,'' Republican editors proclaimed that ``the
movement'' towards procedural codification ``comes from the people, from
the instinctive logic by which an unprejudiced mind grasps the
advantages of the system.''\footnote{``The Code,'' \emph{Weekly
  Standard} (Raleigh), May 26, 1869.}

\hypertarget{conclusion}{%
\subsection{Conclusion}\label{conclusion}}

By addressing our historical questions to a sufficiently large but
narrowly defined corpus of sources, we benefited from a useful symbiosis
of traditional and digital historical methods. Our computational methods
produced useful historical knowledge because they were carefully
tailored to what we knew about the data from traditional historical
work. We knew that code commissioners worked with ``the scissors and
paste-pot,'' as critics complained, and we examined codes in the
archives which showed how commissions literally marked up the
legislation of other states. While we think that one of the most useful
things about digital history is its ability to start with large corpora
and then figure out what was interesting from the past, we have shown
how digital history can also operate by starting with specific
historical questions rather than particular sources. We have shown how a
collection of methods from computer science, including
minhash/locality-sensitive hashing, affinity propagation clustering, and
network analysis, along with the concept of text deformance from
literary studies, can be used to good effect in tracking the changes in
the law, as well as any other discursive field whose texts can be
readily divided into sections. Finally we have shown how it is possible
to work on different scales, using network analysis, visualization, and
algorithmic close reading, and thus to gain both a broad overview of the
law's migration, as well as a highly detailed view of the changes in the
law.

The history of codification on the American periphery challenges
foundational assumptions about American federalism. Scholars commonly
speak of regulating at
\textless80\textgreater\textless9c\textgreater the state
level\textless80\textgreater\textless9d\textgreater{} imagining an
equality between state sovereignties that exists in tension only with
\textless80\textgreater\textless9c\textgreater the federal
level.\textless80\textgreater\textless9d\textgreater{} But the history
of legal practice and civil remedies is one in which the localism
fostered by common law practice rapidly gave way to uniform regulations
promulgated by New York trial lawyers without the slightest interference
of the federal government.\footnote{The equality of the states is a
  foundational assumption in the much-criticized idea of the states as
  laboratories for regulatory experimentation. The
  states-as-laboratories idea emerged from \emph{New State Ice Co.~v.
  Liebmann}, 285 U.S. 262, 311 (1932) (Brandeis, J., dissenting). See
  James A. Gardner, \textless80\textgreater\textless9c\textgreater The
  \textless80\textgreater\textless98\textgreater States-as-Laboratories\textless80\textgreater\textless99\textgreater{}
  Metaphor in State Constitutional
  Law,\textless80\textgreater\textless9d\textgreater{} \emph{Valparaiso
  University Law Review} 30 (1996), 475. For a collection of
  refutations, see Brian Galle \& Joseph Leahy,
  \textless80\textgreater\textless9c\textgreater Laboratories of
  Democracy? Policy Innovation in Decentralized
  Governments,\textless80\textgreater\textless9d\textgreater{}
  \emph{Emory Law Journal} 58 (2009), 1333. Even as federalism scholars
  vigorously refute the idea of states as
  \textless80\textgreater\textless9c\textgreater laboratories\textless80\textgreater\textless9d\textgreater{}
  for regulative experimentation, they continually pose
  \textless80\textgreater\textless9c\textgreater the
  federal\textless80\textgreater\textless9d\textgreater{} to
  \textless80\textgreater\textless9c\textgreater the
  state\textless80\textgreater\textless9d\textgreater{} level, with an
  assumed equality among the numerous sovereignties in the latter
  category. See, for instance, James E. Fleming \& Jacob T. Levy, eds.,
  \emph{Federalism and Subsidiarity} (NYU 2014); Heather Gerken,
  \emph{Beyond Sovereignty, Beyond Autonomy: A
  Nationalist\textless80\textgreater\textless99\textgreater s View of
  Federalism\textless80\textgreater\textless99\textgreater s Future}
  (forthcoming).} The history of the Code also has important
implications for recent scholarship seeking to unearth a long tradition
of \textless80\textgreater\textless9c\textgreater administrative
law\textless80\textgreater\textless9d\textgreater{} among the states
before the twentieth century. These accounts have largely focused on
administrative adjudication or discretionary regulation within a narrow
domain, such as customs houses, but have so far neglected the most
widespread and significant instance of nineteenth-century administrative
lawmaking in America---the spread of remedial codes through
extra-legislative commissions.\footnote{See Daniel Ernst,
  \emph{Tocqueville's Nightmare: The Administrative State Emerges in
  America, 1900--1940} (Oxford: Oxford University Press, 2014); Jerry L.
  Mashaw, \emph{Creating the Administrative Constitution: The Lost One
  Hundred Years of American Administrative Law} (Yale Law Library,
  2012); Gautham Rao, \emph{National Duties: Custom Houses and the
  Making of the American State} (University of Chicago, 2016).} While
these histories have sought to demonstrate that nineteenth-century
Americans could be quite comfortable with administrative law, accepting
it as a normal part of the constitutional order, this chapter has shown
how lawmaking by commission generated significant political controversy
and raised grave questions about popular sovereignty that over time were
merely dodged rather than answered.

In the economically developing West and re-developing South, anxieties
over the lack of capital---both among the bar as well as voters---joined
with arguments about civilization and progress to spur many
jurisdictions to copy the text of the code of New York, the Empire State
of capital. The short legislative sessions of American lawmaking limited
the options available for re-imagining or re-crafting what could become
the law of remedies and legal practice. And in the economically
underdeveloped parts of the country, periods of opportunity could be
short indeed. Capital might quickly pass over one region and favor
another, and each month more lawyers arrived hoping to make a start in a
jurisdiction where economic progress was just about to take off.

This study thus gets at the heart of lawmaking in U.S. history. Lawyers
and judges, politicians and newspaper editors warred over whether codes
that were drafted by commissioners and borrowed wholesale from beyond a
jurisdiction's borders were actually democratic. Codifiers responded by
transmuting democratic theory into support for a remedial code that
elected legislators had neither the time nor inclination to read.
Popular support for commercial development was taken to indicate popular
support for New York's civil remedies, especially the cheapened and
accelerated collection of debts. In many jurisdictions the exact
language of the Field Code remains on the books, and its basic
provisions for civil procedure are in force throughout the United
States. Without too much exaggeration we might say that our method has
revealed the spine of modern American legal practice.

\hypertarget{appendix-clustering-sections-involving-witness-exclusions}{%
\subsection{Appendix: Clustering sections involving witness
exclusions}\label{appendix-clustering-sections-involving-witness-exclusions}}

This sample cluster brings together sections that involve race-based
exclusions from witness testimony in civil trials, with commentary on
the history of the variations.

Disqualifying the testimony of non-white parties and witnesses was not
new to the procedure codes. The following laws from midwestern states
with large free black populations would be echoed in adaptations of the
Field Code.

\begin{longtable}[]{@{}ll@{}}
\toprule
\begin{minipage}[b]{0.17\columnwidth}\raggedright
Code\strut
\end{minipage} & \begin{minipage}[b]{0.77\columnwidth}\raggedright
Section\strut
\end{minipage}\tabularnewline
\midrule
\endhead
\begin{minipage}[t]{0.17\columnwidth}\raggedright
OH 1807\strut
\end{minipage} & \begin{minipage}[t]{0.77\columnwidth}\raggedright
That no black or mulatto person or persons, shall hereafter be permitted
to be sworn or give evidence in any court of record, or elsewhere in
this state, in any cause depending, or matter of controversy, where
either party to the same is a white person, or in any prosecution, which
shall be instituted in behalf of this state, against any white
person.\strut
\end{minipage}\tabularnewline
\begin{minipage}[t]{0.17\columnwidth}\raggedright
\strut
\end{minipage} & \begin{minipage}[t]{0.77\columnwidth}\raggedright
\strut
\end{minipage}\tabularnewline
\begin{minipage}[t]{0.17\columnwidth}\raggedright
IA 1839\strut
\end{minipage} & \begin{minipage}[t]{0.77\columnwidth}\raggedright
A negro, mulatto, or Indian, shall not be a witness in any court or in
any case against a white person.\strut
\end{minipage}\tabularnewline
\begin{minipage}[t]{0.17\columnwidth}\raggedright
\strut
\end{minipage} & \begin{minipage}[t]{0.77\columnwidth}\raggedright
\strut
\end{minipage}\tabularnewline
\begin{minipage}[t]{0.17\columnwidth}\raggedright
IN 1843\strut
\end{minipage} & \begin{minipage}[t]{0.77\columnwidth}\raggedright
No negro, mulatto or Indian, shall be a witness, except in pleas of the
state against negroes, mulattoes, or Indians, and in civil causes where
negroes, mulattoes, or Indians alone are parties: every person other
than a negro having one-fourth part of negro blood or more, or any one
of whose grandfathers or grandmothers shall have been a negro, shall be
deemed an incompetent witness, within the provisions of this
article.\strut
\end{minipage}\tabularnewline
\bottomrule
\end{longtable}

Racial disqualifications were introduced to the Field Code tradition in
California, first in Elisha
Crosby\textless80\textgreater\textless99\textgreater s draft of 1850,
then in Stephen J. Field\textless80\textgreater\textless99\textgreater s
draft of 1851. Many western states copied Stephen
Field\textless80\textgreater\textless99\textgreater s provision.

\begin{longtable}[]{@{}ll@{}}
\toprule
\begin{minipage}[b]{0.17\columnwidth}\raggedright
Code\strut
\end{minipage} & \begin{minipage}[b]{0.77\columnwidth}\raggedright
Section\strut
\end{minipage}\tabularnewline
\midrule
\endhead
\begin{minipage}[t]{0.17\columnwidth}\raggedright
CA 1850\strut
\end{minipage} & \begin{minipage}[t]{0.77\columnwidth}\raggedright
306. No black, or mulatto person, or Indian, shall be permitted to give
evidence in any action to which a white person is a party, in any Court
of this State. Every person who shall have one eighth part or more of
negro blood, shall be deemed a mulatto; and every person who shall have
one half Indian blood, shall be deemed an Indian.\strut
\end{minipage}\tabularnewline
\begin{minipage}[t]{0.17\columnwidth}\raggedright
\strut
\end{minipage} & \begin{minipage}[t]{0.77\columnwidth}\raggedright
\strut
\end{minipage}\tabularnewline
\begin{minipage}[t]{0.17\columnwidth}\raggedright
CA 1851\strut
\end{minipage} & \begin{minipage}[t]{0.77\columnwidth}\raggedright
394. The following persons shall not be witnesses: lst. Those who are of
unsound mind at the time of their production for examination; 2d.
Children under ten years of age, who appear incapable of receiving just
impressions of the facts respecting which they are examined, or of
relating them truly: and; 3d. Indians, or persons having one fourth or
more of Indian blood, in an action or proceeding to which a white person
is a party: 4th. Negroes, or persons having one half or more Negro
blood, in an action or proceeding to which a white person is a
party.\strut
\end{minipage}\tabularnewline
\begin{minipage}[t]{0.17\columnwidth}\raggedright
\strut
\end{minipage} & \begin{minipage}[t]{0.77\columnwidth}\raggedright
\strut
\end{minipage}\tabularnewline
\begin{minipage}[t]{0.17\columnwidth}\raggedright
OR 1854\strut
\end{minipage} & \begin{minipage}[t]{0.77\columnwidth}\raggedright
6. The following persons shall not be competent to testify: 1. Those who
are of unsound mind, or intoxicated at the time of their production for
examination; 2. Children under ten years of age, who appear incapable of
receiving just impressions of the facts respecting which they are
examined, or of relating them truly; 3. Negroes, mulattoes and Indians,
or persons one half or more of Indian blood, in an action or proceeding
to which a white person is a party.\strut
\end{minipage}\tabularnewline
\begin{minipage}[t]{0.17\columnwidth}\raggedright
\strut
\end{minipage} & \begin{minipage}[t]{0.77\columnwidth}\raggedright
\strut
\end{minipage}\tabularnewline
\begin{minipage}[t]{0.17\columnwidth}\raggedright
WA 1855\strut
\end{minipage} & \begin{minipage}[t]{0.77\columnwidth}\raggedright
293. The following persons shall not be competent to testify: 1st. Those
who are of unsound mind, or intoxicated at the time of their production
for examination. 2d. Children under ten years of age, who appear
incapable of receiving just impressions of the facts, respecting which
they are examined, or of relating -them truly. 3d. Indians, or persons
having more than one half Indian blood, in an action or proceeding to
which a white person is a party.\strut
\end{minipage}\tabularnewline
\begin{minipage}[t]{0.17\columnwidth}\raggedright
\strut
\end{minipage} & \begin{minipage}[t]{0.77\columnwidth}\raggedright
\strut
\end{minipage}\tabularnewline
\begin{minipage}[t]{0.17\columnwidth}\raggedright
UT 1859\strut
\end{minipage} & \begin{minipage}[t]{0.77\columnwidth}\raggedright
215. The following persons shall not be competent to testify: 1. Those
who are of unsound mind or intoxicated at the time of their production
for examination. 2. Children under ten years of age, who appear to be
incapable of receiving just impressions of the facts respecting which
they are examined or of relating them truly. Negroes, mulattos, and
Indians, or persons having one fourth of negro or Indian blood, in an
action or proceeding to which a white person is a party, but shall not
be disqualified from testifying against another.\strut
\end{minipage}\tabularnewline
\begin{minipage}[t]{0.17\columnwidth}\raggedright
\strut
\end{minipage} & \begin{minipage}[t]{0.77\columnwidth}\raggedright
\strut
\end{minipage}\tabularnewline
\begin{minipage}[t]{0.17\columnwidth}\raggedright
NV 1861\strut
\end{minipage} & \begin{minipage}[t]{0.77\columnwidth}\raggedright
342. The following persons shall not be witnesses: First. Those who are
of unsound mind at the time of their production for examination. Second.
Children under ten years of age, who, in the opinion of the court,
appear incapable of receiving just impressions of the facts respecting
which they are examined, or of relating them truly. Third. Indians, or
persons having one half or more of Indian blood, and negroes, or persons
having one half or more of negro blood, in an action or proceeding to
which a white person is a party. Fourth. Persons against whom judgment
as been rendered upon a conviction for a felony, unless pardoned by the
governor, or such judgment has been reversed on appeal.\strut
\end{minipage}\tabularnewline
\begin{minipage}[t]{0.17\columnwidth}\raggedright
\strut
\end{minipage} & \begin{minipage}[t]{0.77\columnwidth}\raggedright
\strut
\end{minipage}\tabularnewline
\begin{minipage}[t]{0.17\columnwidth}\raggedright
ID 1864\strut
\end{minipage} & \begin{minipage}[t]{0.77\columnwidth}\raggedright
352. The following persons shall not be witnesses: First. Those who are
of unsound mind at the time of their production for examination. Second.
Children under ten years of age, who, in the opinion of the court,
appear incapable of receiving just impressions of the facts respecting
which they are examined, or of relating them truly. Third. Chinamen or
persons having one-half or more of China blood; Indians, or persons
having one-half or more of Indian blood, and negroes, or persons having
one-half or more of negro blood, in an action or proceeding to which a
white person is a party. Fourth. Persons against whom judgment has been
rendered upon a conviction or felony, unless pardoned by the governor,
or such judgment has been reversed on appeal.\strut
\end{minipage}\tabularnewline
\begin{minipage}[t]{0.17\columnwidth}\raggedright
\strut
\end{minipage} & \begin{minipage}[t]{0.77\columnwidth}\raggedright
\strut
\end{minipage}\tabularnewline
\begin{minipage}[t]{0.17\columnwidth}\raggedright
AZ 1865\strut
\end{minipage} & \begin{minipage}[t]{0.77\columnwidth}\raggedright
396. The following persons shall not be witnesses: 1. Those who are of
unsound mind at the time of their production for examination. 2.
Children under ten years of age, who appear incapable of receiving just
impressions of the facts respecting which they are examined, or of
relating them truly; and, 3. Indiana or persons having one-half or more
of Indian blood, in an action or proceeding to which a white person is a
party. 4. Negroes, or persons having one-half or more negro blood, in an
action or, proceeding to which a white person is a party.\strut
\end{minipage}\tabularnewline
\begin{minipage}[t]{0.17\columnwidth}\raggedright
\strut
\end{minipage} & \begin{minipage}[t]{0.77\columnwidth}\raggedright
\strut
\end{minipage}\tabularnewline
\begin{minipage}[t]{0.17\columnwidth}\raggedright
CA 1868\strut
\end{minipage} & \begin{minipage}[t]{0.77\columnwidth}\raggedright
394. The following persons shall not he witnesses: First. Those who are
of unsound mind at the time of their production for examination. Second.
Children under ten years of age, who, in the opinion of the court,
appear incapable of receiving just impressions of the facts respecting
which they are examined, or of relating them truly. Third. Mongolians,
Chinese, or Indians, or persons having one-half or more of Indian blood,
in an action or proceeding wherein a white person is a party. Fourth.
Persons against whom judgment has been rendered upon a conviction for a
felony, unless pardoned by the governor, or such judgment has been
reversed on appeal.\strut
\end{minipage}\tabularnewline
\bottomrule
\end{longtable}

Kentucky differed from other code states by making no distinction
between incompetency (an absolute bar) and privilege (which might be
waived). Kentucky also maintained a strict disqualification of parties
and interested witnesses while other Field Code states made parties at
least partially competent to stand examination.

\begin{longtable}[]{@{}ll@{}}
\toprule
\begin{minipage}[b]{0.20\columnwidth}\raggedright
Code\strut
\end{minipage} & \begin{minipage}[b]{0.74\columnwidth}\raggedright
Section\strut
\end{minipage}\tabularnewline
\midrule
\endhead
\begin{minipage}[t]{0.20\columnwidth}\raggedright
KY 1851\strut
\end{minipage} & \begin{minipage}[t]{0.74\columnwidth}\raggedright
568. The following persons shall be incompetent to testify: 1. Persons
convicted of a capital offense, or of perjury, subornation of perjury;
burglary, robbery, larceny, receiving stolen goods, forgery, or
counterfeiting. 2. Infants under the age of ten years, and over that
age, if incapable of understanding the obligation of an oath. 3. Persons
who are of unsound mind at the time of being produced as witnesses. 4.
Husband and wife, for or against each other, or concerning any
communication made by one to the other, during the marriage, whether
called as a witness while that relation subsisted or afterwards. 5. An
attorney, concerning any communication made to him by his client in that
relation, or his advice thereon, without the
client\textless80\textgreater\textless99\textgreater s consent. 6.
Persons interested in an issue, in behalf of themselves, and parties to
an issue, in behalf of themselves or those united with them in the
issue. 7. Negroes, mulattoes, or Indians, in any action or proceeding
where a white person, in his own right, or as representative of a white
person, is a party, except in actions brought to recover a penalty or
forfeiture for a violation of law, against a negro, mulatto, or
Indian.\strut
\end{minipage}\tabularnewline
\begin{minipage}[t]{0.20\columnwidth}\raggedright
\strut
\end{minipage} & \begin{minipage}[t]{0.74\columnwidth}\raggedright
\strut
\end{minipage}\tabularnewline
\begin{minipage}[t]{0.20\columnwidth}\raggedright
MT 1865\strut
\end{minipage} & \begin{minipage}[t]{0.74\columnwidth}\raggedright
320. The following persons shall be incompetent to testify: First,
Persons who are of an unsound mind at the time of their production for
examination. Second, Children under ten years of age who appear
incapable of receiving just impressions of the facts respecting which
they are examined or of relating them truly, but the court in its
discretion may allow such children to testify, and the facts herein
enumerated shall go to their credibility. Third, Husband or wife for or
against each other, or concerning any communication made by one to the
other during the marriage, whether called as a witness while that
relation existed or afterwards. Fourth, An attorney concerning any
communication made to him by his client in that relation, or his advice
thereon, without the
client\textless80\textgreater\textless99\textgreater s consent. Fifth, A
clergyman or priest concerning any confession made to him, in his
professional character, in the course of discipline enjoined by the
church to which he belongs, without the consent of the person making the
confession. Sixth, A negro, Indian, or Chinaman, where the parties to
the action are white persons, but if the parties to an action or either
of the parties is an Indian, negro, or Chinaman, a negro may be
introduced as a witness against such negro, an Indian against such
Indian, or a Chinaman against such Chinaman. A negro within the meaning
of this act is a person having one-eighth or more of negro blood, an
Indian is a person having one-half or more of Indian blood, and a
Chinaman is a person having one-half or more Chinese blood.\strut
\end{minipage}\tabularnewline
\bottomrule
\end{longtable}

Iowa's code began by defining competency purely in terms of
understanding the legal oath, and in the same section it barred
non-white testimony (even if non-white actors could understand the
oath). Wyoming followed the same tack, but softened the racial bar by
adopting the same standard used for children: only those adjudged
incapable of perceiving and relating facts were barred from testifying.

\begin{longtable}[]{@{}ll@{}}
\toprule
\begin{minipage}[b]{0.20\columnwidth}\raggedright
Code\strut
\end{minipage} & \begin{minipage}[b]{0.74\columnwidth}\raggedright
Section\strut
\end{minipage}\tabularnewline
\midrule
\endhead
\begin{minipage}[t]{0.20\columnwidth}\raggedright
IA 1851\strut
\end{minipage} & \begin{minipage}[t]{0.74\columnwidth}\raggedright
2388. Every human being of sufficient capacity to understand the
obligation of an oath is a competent witness in all cases both civil and
criminal except as herein otherwise declared. But an indian, a negro, a
mulatto or black person shall not be allowed to give testimony in any
cause wherein a white person is a party.\strut
\end{minipage}\tabularnewline
\begin{minipage}[t]{0.20\columnwidth}\raggedright
\strut
\end{minipage} & \begin{minipage}[t]{0.74\columnwidth}\raggedright
\strut
\end{minipage}\tabularnewline
\begin{minipage}[t]{0.20\columnwidth}\raggedright
NE 1855\strut
\end{minipage} & \begin{minipage}[t]{0.74\columnwidth}\raggedright
2388. Every human being of sufficient capacity to understand the
obligation of an oath is a competent witness in all cases both civil and
criminal except as herein otherwise declared. But an indian, a negro, a
mulatto or black person shall not be allowed to give testimony in any
cause wherein a white person is a party.\strut
\end{minipage}\tabularnewline
\begin{minipage}[t]{0.20\columnwidth}\raggedright
\strut
\end{minipage} & \begin{minipage}[t]{0.74\columnwidth}\raggedright
\strut
\end{minipage}\tabularnewline
\begin{minipage}[t]{0.20\columnwidth}\raggedright
WY 1870\strut
\end{minipage} & \begin{minipage}[t]{0.74\columnwidth}\raggedright
325. Every human being of sufficient capacity to understand the
obligations of an oath, is a competent witness in all cases, civil and
criminal, except as otherwise herein declared. The following persons
shall be incompetent to testify: First, Persons of unsound mind at the
time of their production. Second, Indians and negroes who appear
incapable of receiving just impressions of the facts respecting which
they are examined, or of relating them intelligently and truly. Third
Husband and wife, concerning any communication made by one to the other
during the marriage, whether called as a witness while that relation
exists or afterwards. Fourth, An attorney, concerning any communication
made to him by his client in that relation or his advice thereon,
without the client\textless80\textgreater\textless99\textgreater s
consent in open court or in writing produced in court. Fifth, A
clergyman or priest, concerning any confession made to him in his
professional character in the course of discipline enjoined by the
church to which he belongs, without the consent of the person making the
confession.\strut
\end{minipage}\tabularnewline
\bottomrule
\end{longtable}

States that did not borrow Field's evidence code nevertheless borrowed
prohibitions of non-white testimony. Two distinct strands emerge in
legislation from the Deep South as well as the Upper South and Midwest.

\begin{longtable}[]{@{}ll@{}}
\toprule
\begin{minipage}[b]{0.20\columnwidth}\raggedright
Code\strut
\end{minipage} & \begin{minipage}[b]{0.74\columnwidth}\raggedright
Section\strut
\end{minipage}\tabularnewline
\midrule
\endhead
\begin{minipage}[t]{0.20\columnwidth}\raggedright
MS 1848\strut
\end{minipage} & \begin{minipage}[t]{0.74\columnwidth}\raggedright
All negroes, mulattoes, Indians, and all persons of mixed blood,
descended from negro or Indian ancestors, to the third generation,
inclusive, though one ancestor of each generation may have been a white
person, shall be incapable in law, to be witnesses in any case
whatsoever, except for and against each other.\strut
\end{minipage}\tabularnewline
\begin{minipage}[t]{0.20\columnwidth}\raggedright
\strut
\end{minipage} & \begin{minipage}[t]{0.74\columnwidth}\raggedright
\strut
\end{minipage}\tabularnewline
\begin{minipage}[t]{0.20\columnwidth}\raggedright
AL 1852\strut
\end{minipage} & \begin{minipage}[t]{0.74\columnwidth}\raggedright
2276. Negroes, mulattoes, Indians, and all persons of mixed blood,
descended from negro or Indian ancestors, to the third generation
inclusive, though one ancestor of each generation may have been a white
person, whether bond or e, must not be Witnesses in any cause, civil or
criminal, except for or against each other.\strut
\end{minipage}\tabularnewline
\begin{minipage}[t]{0.20\columnwidth}\raggedright
\strut
\end{minipage} & \begin{minipage}[t]{0.74\columnwidth}\raggedright
\strut
\end{minipage}\tabularnewline
\begin{minipage}[t]{0.20\columnwidth}\raggedright
TN 1858\strut
\end{minipage} & \begin{minipage}[t]{0.74\columnwidth}\raggedright
3808. A negro, mulatto, Indian, or person of mixed blood, descended from
negro or Indian ancestors, to the third generation inclusive, though one
ancestor of each generation may have been a white person, whether bond
or free, is incapable of being a witness in any cause, civil or
criminal, except for or against each other.\strut
\end{minipage}\tabularnewline
\begin{minipage}[t]{0.20\columnwidth}\raggedright
\strut
\end{minipage} & \begin{minipage}[t]{0.74\columnwidth}\raggedright
\strut
\end{minipage}\tabularnewline
\begin{minipage}[t]{0.20\columnwidth}\raggedright
DC 1857\strut
\end{minipage} & \begin{minipage}[t]{0.74\columnwidth}\raggedright
A negro shall be a competent witness for or against a negro in any
criminal proceeding, and shall be a competent witness in any civil case
to which only negroes are parties, but not in any other case.\strut
\end{minipage}\tabularnewline
\begin{minipage}[t]{0.20\columnwidth}\raggedright
\strut
\end{minipage} & \begin{minipage}[t]{0.74\columnwidth}\raggedright
\strut
\end{minipage}\tabularnewline
\begin{minipage}[t]{0.20\columnwidth}\raggedright
VA 1860\strut
\end{minipage} & \begin{minipage}[t]{0.74\columnwidth}\raggedright
A negro or indian shall he a competent witness in a case of the
commonwealth for or against a negro or indian, or in a civil ease to
which only negroes or indians are parties, but not in any other
case.\strut
\end{minipage}\tabularnewline
\begin{minipage}[t]{0.20\columnwidth}\raggedright
\strut
\end{minipage} & \begin{minipage}[t]{0.74\columnwidth}\raggedright
\strut
\end{minipage}\tabularnewline
\begin{minipage}[t]{0.20\columnwidth}\raggedright
IL 1866\strut
\end{minipage} & \begin{minipage}[t]{0.74\columnwidth}\raggedright
A negro, mulatto or Indian shall not be a witness in any court, or in
any case, against a white person. Any person having one-fourth part
negro blood shall be adjudged a mulatto.\strut
\end{minipage}\tabularnewline
\bottomrule
\end{longtable}


\end{document}
